\graphicspath{{01_calcolo_in_piu_variabili/figures/PNG/}{01_calcolo_in_piu_variabili/figures/PDF/}{01_calcolo_in_piu_variabili/figures/}}

\chapter{Calcolo in più variabili}
\copyrightnotice
\section{Metrica (euclidea) su $\mathbb{R}^2$ e Topologia}
\begin{definition}
Si chiama \emph{metrica euclidea su $\mathbb{R}^2$} (o anche \emph{distanza euclidea su $\mathbb{R}^2$}) la funzione
$$\mathrm{d}:\mathbb{R}^2 \times \mathbb{R}^2 \longrightarrow \left[0,\, +\infty\right)$$
definita da:
$$\mathrm{d}(P,\, Q) \overset{def}{=} \sqrt{(x_2-x_1)^2+(y_2-y_1)^2}$$
se $P=(x_1,\, y_1),Q=(x_2,\, y_2) \in \mathbb{R}^2$.
\end{definition}

\begin{exer}
Mostrare che la distanza $\mathrm{d}$ verifica le seguenti proprietà:
\begin{enumerate}[labelindent=\parindent,leftmargin=*,label=\textnormal{(d\arabic*)},start=1]
\item $\mathrm{d}(P,\, Q) = 0 \Longleftrightarrow P=Q$
\item $\mathrm{d}(P,\, Q) = \mathrm{d}(Q,\, P) \qquad \forall \, P,Q \in \mathbb{R}^2$
\item (Disuguaglianza triangolare) $\mathrm{d}(P,\, Q) \leq \mathrm{d}(P,\, R) + \mathrm{d}(R,\, Q) \qquad \forall \, P,Q,R \in \mathbb{R}^2$
\end{enumerate}
\end{exer}

Più in generale, dato un (qualunque) insieme $X$, diamo la seguente definizione.
\begin{definition}
Si chiama \emph{distanza} (o \emph{metrica}) su $X$ una funzione
$$\mathrm{d}:X \times X \longrightarrow \left[0,\, +\infty\right)$$
verificante le proprietà \textnormal{(d1)}, \textnormal{(d2)} e \textnormal{(d3)}.
\end{definition}

\begin{definition}
Dato $P_0 \in \mathbb{R}^2$, $P_0 = (x_0, y_0)$ fissato, e dato $r>0$ fissato, si chiama \emph{intorno (sferico)} di centro $P_0$ e raggio $r>0$ (in $\mathbb{R}^2$) l'insieme:
\begin{center}
\begin{align*}
\underbrace{\mathrm{I}(P_0,\, r)}_{\text{\tiny (Notazione italiana)}} =
\overbrace{\mathrm{B}(P_0,\, r)}^{\text{\tiny (Notazione internazionale, ``Ball'')}} &=
\lbrace P \in \mathbb{R}^2 : \mathrm{d}(P,\, P_0) < r \rbrace =\\
&= \lbrace (x,\, y) \in \mathbb{R}^2 : (x-x_0)^2+(y-y_0)^2 < r^2 \rbrace
\end{align*}
\end{center}
Detto in modo spicciolo, è ``il cerchio di centro $P_0$ e raggio $r$ a cui tagliate la buccia'' \textnormal{[cit. prof]}.
\end{definition}

\begin{center}
\def\svgwidth{6cm}
\input{./01_calcolo_in_piu_variabili/figures/ball.pdf_tex}
\end{center}

\begin{definition}
Sia $A \subseteq \mathbb{R}^2$.
\begin{enumerate}[labelindent=\parindent,leftmargin=*,label=\textnormal{(\roman*)},start=1]
\item Un punto $P_0 \in \mathbb{R}^2$ si dice \emph{punto di accumulazione di $A$} se:
$$(A \setminus \lbrace P_0 \rbrace) \cap \mathrm{B}(P_0,\, r) \neq \emptyset \qquad \forall \, r>0$$
\item Un punto $P_0 \in \mathbb{R}^2$ si dice \emph{punto isolato di $A$} se:
$$\exists \, r_0 > 0 : A \cap \mathrm{B}(P_0,\, r_0) = \lbrace P_0 \rbrace$$
\item Un punto $P_0 \in \mathbb{R}^2$ si dice \emph{punto di frontiera di $A$} se:
$$A \cap \mathrm{B}(P_0,\, r) \neq \emptyset \qquad \text{e} \qquad (\mathbb{R}^2 \setminus A) \cap \mathrm{B}(P_0,\, r) \neq \emptyset, \qquad \forall \, r>0$$
L'insieme dei punti di frontiera di $A$ si chiama \emph{frontiera di $A$} e si denota col simbolo $\partial A$.
\item Un punto $P_0 \in \mathbb{R}^2$ si dice \emph{punto interno di $A$} se:
$$\exists \, r_0 > 0 : \mathrm{B}(P_0,\, r_0) \subseteq A$$
L'insieme di (tutti) i punti interni di $A$ si chiama \emph{parte interna di $A$} e si denota con \AA.
\item L'insieme $A$ si dice \emph{insieme aperto} se $A=\text{\AA}$.
\item L'insieme $A$ si dice \emph{insieme chiuso} se $A \supseteq \partial A$.
\item Si dice \emph{chiusura} di $A$ e si denota con $\overline{A} = A \cup \partial A$.
\end{enumerate}
\end{definition}

\begin{center}
\def\svgwidth{10cm}
\input{./01_calcolo_in_piu_variabili/figures/punti.pdf_tex}
\end{center}

\begin{obs}
Dato $A \subset \mathbb{R}^2$, vale che:
$$A\text{ è aperto} \qquad \Longleftrightarrow \overbrace{\mathbb{R}^2 \setminus A}^{\text{\tiny Complementare di A}} \text{ è chiuso}$$
Infatti l'uguaglianza segue osservando le seguenti due proprietà:
\begin{enumerate}[labelindent=\parindent,leftmargin=*,label=\textnormal{(\arabic*)},start=1]
\item $\partial A = \partial (\mathbb{R}^2 \setminus A)$
\item $A\text{ è aperto} \quad \Longleftrightarrow \quad A \cap \partial A = \emptyset$
\end{enumerate}
\end{obs}

\section{Successioni e topologia di $\mathbb{R}^2$}
\begin{definition}
Si chiama \emph{successione di $\mathbb{R}^2$} una funzione:
$$P:\mathbb{N} \longrightarrow \mathbb{R}^2$$
dove
$$P(h)=P_h \qquad (h=1,2,\ldots )$$
e $P_h$ si chiama \emph{termine h-esimo} della successione $P$.
Tipicamente una successione di $\mathbb{R}^2$ si denota con
$$\lbrace P_h \rbrace_{h \in \mathbb{N}} \qquad \text{o con} \qquad (P_h)_{h \in \mathbb{N}}$$
\end{definition}

Osserviamo che, data $\overbrace{(P_h)_h}^{\text{``}h \in \mathbb{N}\text{''} \text{si omette}} \subset \quad \mathbb{R}^2$, allora $P_h = (x_h,\, y_h)$, con $(x_h)_h,\, (y_h)_h \subset \mathbb{R}$ successioni di numeri reali.

\begin{definition}[limite di una successione in $\mathbb{R}^2$]
Siano $(P_h)_h \subset \mathbb{R}^2$ e $P_0 \in \mathbb{R}^2$. Si dice che la successione $(P_h)_h$ ammette limite $P_0$ (in $\mathbb{R}^2$), e si scrive che $\exists \, \displaystyle \lim_{h \rightarrow +\infty} P_h = P_0$  (in $\mathbb{R}^2$) se
$$\exists \, \lim_{h \rightarrow +\infty} \mathrm{d}(P_h,\, P_0)=0 \qquad \text{(in }\mathbb{R}\text{)}$$
cioè
$$\forall \, \varepsilon > 0 \quad \exists \, \overline{h} = \overline{h}(\varepsilon) \in \mathbb{N} : \mathrm{d}(P_h,\, P_0) < \varepsilon, \qquad \forall \, h \overset{\tiny (\geq)}{>} \overline{h}$$
\end{definition}

Ma\dots operativamente? Ricordiamo innanzitutto che, presi $P_h(x_h,\, y_h)$ e $P_0(x_0,\, y_0)$, allora $\mathrm{d}(P_h,\, P_0) = \sqrt{(x_h-x_0)^2-(y_h-y_0)^2}$. Ci avvaliamo quindi del risultato del seguente teorema.

\begin{thm}
Siano $(P_h)_h \subset \mathbb{R}^2$ e $P_0 \in \mathbb{R}^2$.
\begin{enumerate}[labelindent=\parindent,leftmargin=*,label=\textnormal{(\roman*)},start=1]
\item Se $P_h=(x_h,\, y_h)$ e $P_0=(x_0,\, y_0)$, allora vale che:
$$\exists \, \lim_{h \rightarrow +\infty} P_h = P_0 \; \text{(in } \mathbb{R}^2\text{)} \quad \Longleftrightarrow \quad 
\begin{cases}
\exists \, \lim_{h \rightarrow +\infty} x_h = x_0\\
\exists \, \lim_{h \rightarrow +\infty} y_h = y_0
\end{cases}
 \; \text{(in } \mathbb{R}\text{)}$$
\item (Criterio di Cauchy)
$$\exists \, \lim_{h \rightarrow +\infty} P_h = P_0 \; \text{(in } \mathbb{R}^2\text{)}$$
$$\Updownarrow$$
$$\forall \, \varepsilon > 0 \quad \exists \, \overline{h}=\overline{h}(\varepsilon) \in \mathbb{N} \quad \text{tale che} \quad \mathrm{d}(P_h,\, P_k) < \varepsilon \quad \forall h,k > \overline{h}$$
\end{enumerate}
\end{thm}
\begin{proof}
\mbox{}
\begin{enumerate}[labelindent=\parindent,leftmargin=*,label=\textnormal{(\roman*)},start=1]
\item Ricordiamo che, dati $a,\, b \in \mathbb{R}$, allora
\begin{center}
\hfill
$\displaystyle \frac{|a|+|b|}{\sqrt{2}} \leq \sqrt{a^2+b^2} \leq |a|+|b|$
\hfill\refstepcounter{equation}\textnormal{(\theequation)}
\end{center}
Scegliendo come $a=x_h-x_0$ e $b=y_h-y_0$ otteniamo che:
\begin{center}
\hfill
$\displaystyle \frac{|x_h-x_0|+|y_h-y_0|}{\sqrt{2}} \leq \mathrm{d}(P_h-P_0) \leq |x_h-x_0|+|y_h-y_0|$
\hfill\refstepcounter{equation}\textnormal{(\theequation)}
\end{center}

Usando il criterio del confronto per successioni di numeri reali su \textnormal{(1.2)}, segue la tesi (in tutti e due i versi).

\item Il verso \textnormal{(}$\Rightarrow$\textnormal{)} è banale, perchè se $\mathrm{d}(P_h,\, P_k) < \varepsilon$, allora anche $|x_h-x_k| < \varepsilon$ e $|y_h-y_k| < \varepsilon$, quindi sono successioni di Cauchy.

\textnormal{(}$\Leftarrow$\textnormal{)} Scegliendo $a=x_h-x_k$ e $b=y_h-y_k$, dalla \textnormal{(1.1)} segue che:
\begin{center}
\hfill
$\displaystyle \frac{|x_h-x_k|+|y_h-y_k|}{\sqrt{2}} \leq \mathrm{d}(P_h-P_k) \leq |x_h-x_k|+|y_h-y_k| \qquad \forall \, h,\, k \in \mathbb{N}$
\hfill\refstepcounter{equation}\textnormal{(\theequation)}
\end{center}
Poichè $(x_h)_h,\; (y_h)_h \subset \mathbb{R}$, applicando la \textnormal{(1.3)} e il Criterio di Cauchy per successioni di numeri reali, segue la tesi.

\end{enumerate}
\end{proof}

\begin{obs}
Dal punto $(i)$ del teorema precedente e dall'unicità del limite per successioni di numeri reali, segue l'unicità del limite per successioni di $\mathbb{R}^2$.
\end{obs}

\begin{proposition}
Sia $A \subset \mathbb{R}^2$.
\begin{enumerate}[labelindent=\parindent,leftmargin=*,label=\textnormal{(\roman*)},start=1]
\item $P_0 \in \mathbb{R}^2$ è un punto di accumulazione per $A$ se e solo se $\exists \, (P_h)_h \subset A \setminus \lbrace P_0 \rbrace$ tale che $\exists \, \displaystyle \lim_{h \rightarrow +\infty} P_h = P_0$.
\item $A$ è chiuso$\quad \Longleftrightarrow \quad$Per ogni successione $(P_h)_h \subset A$ con $\displaystyle \lim_{h \rightarrow +\infty} P_h = P_0 \in \mathbb{R}^2 \quad \Rightarrow \quad P_0 \in A$.
\end{enumerate}
\end{proposition}
\begin{proof}
(Vedi Esercizio 4 del Foglio 1).
\end{proof}

\begin{definition}
Siano $A \subseteq \mathbb{R}^2$, $P_0 \in \mathbb{R}^2$ fissato, con $P_0=(x_0,\, y_0)$ punto di accumulazione di $A$. Sia $f : A \longrightarrow \mathbb{R}$, e sia $L \in \mathbb{R}$.
Si dice che
\begin{center}
$\displaystyle \lim_{P \rightarrow P_0} f(P) = L \qquad$
o, equivalentemente,
$\displaystyle \qquad \lim_{(x,\, y) \rightarrow (x_0,\, y_0)} f(x,\, y) = L$
\end{center}
se, per definizione, $\forall \, \varepsilon > 0 \quad \exists \, \delta = \delta(P_0,\, \varepsilon) > 0$ tale che
$$|f(P) - L| < \varepsilon \qquad \forall \, P \in (A \setminus \lbrace P_0 \rbrace) \cap B(P_0,\, \delta)$$
\end{definition}

\begin{thm}[Caratterizzazione del limite di una funzione tramite il limite di successioni]
Siano $A \subseteq \mathbb{R}^2$, $P_0 \in \mathbb{R}^2$ fissato, con $P_0=(x_0,\, y_0)$ punto di accumulazione di $A$. Sia $f : A \longrightarrow \mathbb{R}$, e sia $L \in \mathbb{R}$. Allora
\begin{enumerate}[labelindent=\parindent,leftmargin=*,label=\textnormal{(\arabic*)},start=1]
\item \hfill $\displaystyle \exists \, \lim_{P \rightarrow P_0} f(P) = L$ \hfill \null
$$\Updownarrow$$
\item
\hfill
$\displaystyle \forall \, (P_h)_h \subset A \setminus \lbrace P_0 \rbrace \; \text{ con} \; \lim_{h \rightarrow +\infty} P_h = P_0 \; \text{(in } \mathbb{R}^2 \text{)} \; \Longrightarrow \; \exists \, \lim_{h \rightarrow +\infty} f(P_h) = L \; \text{(in } \mathbb{R} \text{)}$
\hfill \null
\end{enumerate}
\begin{center}
\def\svgwidth{15cm}
\input{./01_calcolo_in_piu_variabili/figures/limite_succ.pdf_tex}
\end{center}

\end{thm}
\begin{proof}
Il verso $(\Rightarrow)$ è ovvio, per la definizione di limite di una successione reale.

Dimostriamo il verso $(\Leftarrow)$. Procediamo per \emph{assurdo}, ossia supponiamo che
$$\nexists \, \lim_{P \rightarrow P_0} f(P)$$
In altri termini,
$$\exists \, \varepsilon_0 > 0 \; : \; \forall \, \delta > 0 \quad \exists \, P_{\delta} \in (A \setminus \lbrace P_0 \rbrace) \cap B(P_0,\, \delta) \quad \text{tale che} \quad |f(P_{\delta}) - L| \geq \varepsilon_0$$
Potendo scegliere $\delta$ in modo arbitrario, scegliamo $\delta=\frac{1}{h}$, dove $h \in \mathbb{N}_{\geq 1}$. In corrispondenza di tale $\delta$, esiste un $P_{\frac{1}{h}} \equiv P_h \in (A \setminus \lbrace P_0 \rbrace) \cap B(P_0,\, \frac{1}{h})$ tale che
\begin{center}
$\mathrm{(3)}$
\hfill
$\displaystyle |f(P_h) - L| \geq \varepsilon_0 > 0$
\hfill \null
\end{center}
Pertanto, abbiamo costruito una successione $(P_h)_h$ tale che $0 \leq \mathrm{d}(P_h,\, P_0) < \frac{1}{h}$ e dunque
$$\lim_{h \rightarrow +\infty} \mathrm{d}(P_h,\, P_0) = 0 \qquad \overset{\text{def.}}{\Longleftrightarrow} \qquad \mathrm{(4)} \quad \exists \lim_{h \rightarrow +\infty} P_h = P_0 \; \text{(in } \mathbb{R}^2 \text{)}$$
\begin{center}
\def\svgwidth{6cm}
\input{./01_calcolo_in_piu_variabili/figures/p_h.pdf_tex}
\end{center}
Per ipotesi, questo significa che
$$\exists \, \lim_{h \rightarrow +\infty} f(P_h) = L$$
ossia, utilizzando la definizione, 
$$\forall \, \varepsilon > 0 \quad \exists \, \overline{h} = \overline{h}(\varepsilon) \in \mathbb{N} : |f(P_h) - L| < \varepsilon, \qquad \forall \, h > \overline{h}$$
In particolare, abbiamo che
$$|f(P_h) - L| < \varepsilon_0$$
in contraddizione con \textnormal{(3)}.
\end{proof}

Come corollario di questo teorema, otteniamo il seguente risultato.

\begin{cor}[Unicità del limite]
Siano $A \subseteq \mathbb{R}^2$, $P_0 \in \mathbb{R}^2$ fissato, con $P_0=(x_0,\, y_0)$ punto di accumulazione di $A$. Sia $f : A \longrightarrow \mathbb{R}$, e siano $L_1,\, L_2 \in \mathbb{R}$ tali che
$$
\left.
\begin{array}{c}
\displaystyle \exists \, \lim_{P \rightarrow P_0} f(P) = L_1 \\ 
e \\
\displaystyle \exists \, \lim_{P \rightarrow P_0} f(P) = L_2
\end{array}
\right\rbrace
\quad \Longrightarrow \quad L_1 = L_2
$$
\end{cor}
\begin{proof}
La dimostrazione segue dal teorema precedente e dall'unicità del limite per successioni di numeri reali.
\end{proof}

\begin{thm}[Operazioni per limiti di funzioni di due variabili]
Siano $f,\, g : A \longrightarrow \mathbb{R}$, dove $A \subseteq \mathbb{R}^2$, $P_0 \in \mathbb{R}^2$ punto di accumulazione per $A$. Supponiamo che $\displaystyle \exists \, \lim_{P \rightarrow P_0} f(P) = L \in \mathbb{R}$, e che $\displaystyle \exists \, \lim_{P \rightarrow P_0} g(P) = M \in \mathbb{R}$. Allora:
\begin{enumerate}[labelindent=\parindent,leftmargin=*,label=\textnormal{(\roman*)},start=1]
\item $\displaystyle \exists \, \lim_{P \rightarrow P_0} (f + g)(P) = L + M$
\item $\displaystyle \exists \, \lim_{P \rightarrow P_0} (f \cdot g)(P) = L \cdot M$
\item Se $\quad g(P) \neq 0 \quad \forall \, P \in A$, e $M \neq 0, \quad \text{allora} \displaystyle \quad \exists \, \lim_{P \rightarrow P_0} \left( \frac{f}{g} \right) (P) = \frac{L}{M}$
\item Supponiamo che $\; f(P) \leq g(P) \quad \forall \, P \in A \setminus \lbrace P_0 \rbrace$. Allora $\; L \leq M$.
\item Sia $\; F: \mathbb{R} \longrightarrow \mathbb{R} \;$ continua, allora $\displaystyle \; \exists \, \lim_{P \rightarrow P_0} (F \circ f)(P) = F(L)$
\end{enumerate}
\end{thm}
\begin{proof}
\begin{enumerate}[labelindent=\parindent,leftmargin=*,label=\textnormal{(\roman*)},start=1]
\item Usando il teorema sulla caratterizzazione del limite tramite successioni, per ipotesi abbiamo che $\displaystyle \exists \lim_{h \longrightarrow +\infty} f(P_h) = L \quad \text{e} \quad \exists \lim_{h \longrightarrow +\infty} g(P_h) = M$. Definendo $(a_h)_h \doteqdot (f(P_h))_h \quad \text{e} \quad (b_h)_h \doteqdot (g(P_h))_h$, sappiamo che $(a_h)_h \; \text{e} \; (b_h)_h$ hanno limite finito rispettivamente $L \; \text{e} \; M$ se e solo se (rispettivamente) $(a_h - L)_h \; \text{e} \; (b_h - M)_h$ sono infinitesime, che a loro volta sono infinitesime se e solo se $(|a_h - L|)_h \; \text{e} \; (|b_h - M|)_h$ sono infinitesime. Poiché una somma di successioni infinitesime è ancora infinitesima, abbiamo che $(|a_h - L| + |b_h - M|)_h$ è infinitesima. Aggiungendo a questo fatto la disuguaglianza
$$|a_h + b_h - (L + M)| \leq |a_h - L| + |b_h - M|$$
otteniamo che $(a_h + b_h - (L + M))_h$ è infinitesima, ossia $\; \displaystyle \lim_{P \rightarrow P_0} (f + g)(P) = L + M$.
\item Si procede come nella dimostrazione (i), utilizzando il fatto che il prodotto di una successione infinitesima per una successione limitata è ancora infinitesima e le disuguaglianze
$$|a_h b_h - LM| \leq |a_hb_h - a_hM| + |a_hM - LM| \leq |a_h| \cdot |b_h - M| + |M| \cdot |a_h - L|$$
\item Si procede come nella dimostrazione (i), utilizzando il fatto che il prodotto di una successione infinitesima per una successione limitata è ancora infinitesima, l'uguaglianza
$$|\frac{1}{b_h} - \frac{1}{M}| = \frac{1}{|Mb_h|} \cdot |b_h - M|$$
e il fatto che la successione $\left( \frac{1}{|Mb_h|} \right)_h$ è limitata.
\item Poiché $f(P) \leq g(P) \quad \forall \, P \in A \setminus \lbrace P_0 \rbrace$, possiamo supporre che $\; \exists \, \lambda_{f(P)} = \lambda (f(P)) > 0$ tale che $f(P) + \lambda_{f(P)} = g(P) \quad \forall \, P \in A \setminus \lbrace P_0 \rbrace$.

Passando a limite e utilizzando (i), otteniamo che
$$ \lim_{P \rightarrow P_0} (f(P) + \lambda_{f(P)}) = \lim_{P \rightarrow P_0} f(P) \quad \Longleftrightarrow \quad \lim_{P \rightarrow P_0} (f(P)) + \lim_{P \rightarrow P_0} (\lambda_{f(P)}) = \lim_{P \rightarrow P_0} f(P)$$
ossia $L \leq M$.
\item Proviamo che, per ogni successione $(P_h)_h \subset A \setminus \lbrace P_0 \rbrace \;$ con $\; \displaystyle \lim_{h \rightarrow +\infty} P_h = P_0$, vale che
$$\exists \, \lim_{h \rightarrow +\infty} F(f(P_h)) = F(L) \qquad \text{(in } \mathbb{R} \text{)}$$
Sia $(P_h)_h \subset A \setminus \lbrace P_0 \rbrace \;$ tale che $\; \displaystyle \lim_{h \rightarrow +\infty} P_h = P_0$. Per ipotesi, se denotiamo con $a_h \doteqdot f(P_h) \quad (h \in \mathbb{N})$, allora per $(1) \Longrightarrow (2)$ del teorema sulla caratterizzazione del limite, $\; \displaystyle \lim_{h \rightarrow +\infty} a_h = L \quad \text{(in } \mathbb{R} \text{)}$.

Consideriamo la successione $(F(a_h))_h \subset \mathbb{R}$. Poiché $F$ è continua nel punto $L$, per definizione di funzione continua su $\mathbb{R}$ segue che $\displaystyle \exists \, lim_{h \rightarrow +\infty} F(a_h) = F(L)$. La tesi è dimostrata.
\end{enumerate}
\end{proof}

\begin{cor}[Teorema dei due carabinieri]
Siano $\; f,\, g,\, h : A \longrightarrow \mathbb{R} \;$ con $\; A \subseteq \mathbb{R}^2, \; P_0 \in \mathbb{R}^2 \;$ punto di accumulazione per $A$. Supponiamo che
\begin{enumerate}[labelindent=\parindent,leftmargin=*,label=\textnormal{(\roman*)},start=1]
\item $g(P) \leq f(P) \leq h(P) \qquad \forall \, P \in A \setminus \lbrace P_0 \rbrace$
\item $\exists \, \displaystyle \lim_{P \rightarrow P_0} g(P) = \lim_{P \rightarrow P_0} h(P) = L \in \mathbb{R}$
\end{enumerate}
Allora
$$\exists \, \lim_{P \rightarrow P_0} f(P) = L$$
\end{cor}

\begin{example}[i]
Verificare che
$$\lim_{(x,\, y) \rightarrow (0,\, 0)} \frac{\sin(x^2+y^2)}{x^2+y^2} = 1$$
\end{example}
\begin{proof}
Definiamo $\; h(x,\, y) : \mathbb{R}^2 \setminus \lbrace (0,\, 0) \rbrace \longrightarrow \mathbb{R} \;$ tale che
$$h(x,\, y) = \frac{\sin(x^2+y^2)}{x^2+y^2}$$
Definiamo poi $\; f(x,\, y) : \mathbb{R}^2 \setminus \lbrace (0,\, 0) \rbrace \longrightarrow \mathbb{R} \;$ tale che $\; f(x,\, y) = x^2 + y^2 \;$ e $\; F : \mathbb{R} \longrightarrow \mathbb{R}$ tale che
$$
F(t) \doteqdot
\begin{cases}
\frac{\sin (t)}{t} & t \neq 0 \\
1 & t = 0
\end{cases}
$$
\`E ovvio che $\; \displaystyle \exists \, \lim_{(x,\, y) \rightarrow (0,\, 0)} f(x,\, y) = 0$. D'altra parte, $F$ è continua in $\mathbb{R}$ (per definizione). Applicando il punto (v) del precedente teorema, otteniamo che
$$\exists \, \lim_{(x,\, y) \rightarrow (0,\, 0)} h(x,\, y) = F(0) = 1$$
\end{proof}

\begin{example}[ii]
Calcolare, se esiste,
$$\lim_{(x,\, y) \rightarrow (0,\, 0)} \frac{xy}{x^2+y^2}$$
\end{example}
\begin{proof}
Dimostriamo che tale limite non esiste.

Il limite, se esiste, deve essere uguale da tutte le direzioni. Non solo dalle ``direzioni'' formate da rette, ma anche da quelle (ad esempio) formate da parabole. Scegliamo quindi la ``direzione'' $y = mx$ e restringiamo $f$ a tale retta, cioè consideriamo $\; f(x,\, mx) = \frac{x(mx)}{x^2+(mx)^2} = \frac{mx^2}{x^2(1 + m^2)} = \frac{m}{1 + m^2}, \quad x \neq 0$. Sulla retta $\; y=x \; (m=1) \;$ scegliamo la successione $\; P_h = (\frac{1}{h},\, \frac{1}{h}), \; h \in \mathbb{N}_{\geq 1} \;$, mentre sulla retta $\; y=2x \; (m=2) \;$ scegliamo la successione $\; Q_h = (\frac{1}{h},\, \frac{2}{h}), \; h \in \mathbb{N}_{\geq 1}$. \`E immediato verificare che
$$\exists \, \lim_{h \rightarrow +\infty} P_h = \lim_{h \rightarrow +\infty} Q_h = (0,\, 0)$$
D'altra parte, $\; f(P_h) = \frac{1}{2} \quad \forall \, h \in \mathbb{N}$, e $\; f(Q_h) = \frac{2}{5} \quad \forall \, h \in \mathbb{N}$, da cui
$$
\exists \, \lim_{h \rightarrow +\infty} f(P_h) = \frac{1}{2} \neq \frac{2}{5} = \lim_{h \rightarrow +\infty} f(Q_h)
$$
$$\Downarrow$$
$$
\nexists \, \lim_{(x,\, y) \rightarrow (0,\, 0)} \frac{xy}{x^2+y^2}
$$
\end{proof}



\section{Estensione della metrica e topologia a $\mathbb{R}^n$ con $n \geq 2$}
\begin{definition}
Si definisce \emph{distanza (euclidea)} su $\mathbb{R}^n$ la funzione$$\mathrm{d}:\mathbb{R}^n \times \mathbb{R}^n \longrightarrow \left[0,\, +\infty\right)$$
definita da:
$$\mathrm{d}(P,\, Q) \overset{def}{=} \sqrt{(x_1-y_1)^2+(x_2-y_2)^2+ \ldots +(x_n-y_n)^2}$$
se $P=(x_1,\, x_2,\, \ldots \, ,x_n),Q=(y_1,\, y_2,\, \ldots \, ,y_n) \in \mathbb{R}^n$.
\end{definition}

\begin{exer}
Mostrare che la distanza $\mathrm{d}$ verifica le seguenti proprietà:
\begin{enumerate}[labelindent=\parindent,leftmargin=*,label=\textnormal{(d\arabic*)},start=1]
\item $\mathrm{d}(P,\, Q) = 0 \Longleftrightarrow P=Q$
\item $\mathrm{d}(P,\, Q) = \mathrm{d}(Q,\, P) \qquad \forall \, P,Q \in \mathbb{R}^n$
\item (Disuguaglianza triangolare) $\mathrm{d}(P,\, Q) \leq \mathrm{d}(P,\, R) + \mathrm{d}(R,\, Q) \qquad \forall \, P,Q,R \in \mathbb{R}^n$
\end{enumerate}
\end{exer}

\begin{definition}
Dato $\; P_0 = (x_1^0,\ldots,x_n^0) \in \mathbb{R}^n \;$ e $r>0$, si chiama \emph{intorno (sferico)} di centro $P_0$ e raggio $r>0$ l'insieme definito da:
$$\mathrm{B}(P_0,\, r)=
\lbrace P \in \mathbb{R}^n : \mathrm{d}(P,\, P_0) < r \rbrace = \lbrace (x_1,\ldots,x_n) \in \mathbb{R}^n : (x_1-x_1^0)^2+(x_n-x_n^0)^2 < r^2 \rbrace$$
\end{definition}

In generale, possiamo dire che le nozioni topologiche introdotte per $n=2$ e la definizione di limite per funzioni a due variabili si estendono al caso $n \geq 3$.

Riflettiamo ora un momento su un'implicita identificazione che si sta facendo, ossia quella tra punti e vettori. Ad esempio, $P = (x,\, y)$ è un punto di $\mathbb{R}^2$, ma si scrive spesso anche $\underline{P} = \overset{\rightarrow}{P} = (x,\, y)$, intendendolo come vettore.

Alle luce di questo, diamo la seguente definizione. 

\begin{definition}[di norma in $\mathbb{R}^n$]
Si chiama \emph{norma su $\mathbb{R}^n$} una funzione $$\mathrm{N}:\mathbb{R}^n \longrightarrow \left[0,\, +\infty\right)$$
verificante le seguenti proprietà:
\begin{enumerate}[labelindent=\parindent,leftmargin=*,label=\textnormal{(N\arabic*)},start=1]
\item $\mathrm{N}(P)=0 \quad \Longleftrightarrow \quad P=\underline{0}=(0,\ldots,0) \in \mathbb{R}^n$
\item $\mathrm{N}(\lambda P) = |\lambda|\mathrm{N}(P) \qquad \forall \, \lambda \in \mathbb{R}, \; \forall \, P \in \mathbb{R}^n$
\item (Disuguaglianza triangolare) $\mathrm{N}(P+Q) \leq \mathrm{N}(P) + \mathrm{N}(Q) \qquad \forall \, P,\, Q \in \mathbb{R}^n$
\end{enumerate}
\begin{center}
\def\svgwidth{7cm}
\input{./01_calcolo_in_piu_variabili/figures/somma_vettori.pdf_tex}
\end{center}
\end{definition}

\begin{definition}[di norma euclidea]
Si chiama \emph{norma (euclidea)} in $\mathbb{R}^n$ la funzione $\; ||\cdot||:\mathbb{R}^n \longrightarrow \left[0,\, +\infty \right) \;$ tale che
$$||P|| = \sqrt{x_1^2+x_2^2+\ldots+x_n^2} \qquad \quad \text{se} \quad P=(x_1,\ldots,x_n) \in \mathbb{R}^n$$
\end{definition}

\begin{obs}
\mbox{}
\begin{enumerate}[labelindent=\parindent,leftmargin=*,label=\textnormal{(\roman*)},start=1]
\item $\mathrm{d}(P,\, Q) = ||P-Q|| \qquad \forall \, P,\, Q,\, R \in \mathbb{R}^n$.

In particolare, da quest'osservazione discende che:
$$\mathrm{d}(P+R,\, Q+R) = \mathrm{d}(P,\, Q)$$

\item Notiamo che $||P|| = \sqrt{\underline{P} \bullet \underline{P}}$ dove, per definizione, dati due vettori $P,\, Q \in \mathbb{R}^n$,
$$P \bullet Q = P^TQ = \sum_{i=1}^{n} x_iy_i$$
se $P=(x_1,\ldots,x_n)$ e $Q=(y_1,\ldots,y_n)$. Poiché questa definizione è stata data inizialmente dai fisici, ricordiamo che:
$$P \bullet Q = ||P||\cdot||Q||\cos(\vartheta)$$
dove $\vartheta$ è l'angolo compreso fra i due vettori.

\item (disuguaglianza di Cauchy-Schwarz)
$$|P \bullet Q| \leq ||P||\cdot||Q|| \qquad \forall \, P,\, Q \in \mathbb{R}^n$$
(Ex. 11 Foglio 1). Provare che da (C-S) segue la proprietà (N3), e dunque la (d3).

Il risultato (ottenuto peraltro da Riemann) è notevolissimo, poiché ci consente di affermare che basta un prodotto scalare per indurre una distanza!
\end{enumerate}
\end{obs}



\section{Funzioni continue di più variabili (a valori reali)}
\begin{definition}
Siano $A \subseteq \mathbb{R}^n$, $f:A \longrightarrow \mathbb{R}$, $P_0 \in A$.
\begin{enumerate}[labelindent=\parindent,leftmargin=*,label=\textnormal{(\roman*)},start=1]
\item Si dice che \emph{$f$ è continua nel punto $P_0 \in A$} se
	\begin{enumerate}[labelindent=\parindent,leftmargin=*,label=\textnormal{(\arabic*)},start=1]
	\item $P_0$ è un \emph{punto isolato} di $A$, cioè, per definizione, $\; \exists \, r_0>0 \quad \text{tale che} \quad \break A \cap \mathrm{B}(P_0,\, r_0) = \lbrace P_0 \rbrace$
	
	\begin{center}oppure\end{center}
	
	\item $P_0$ è un \emph{punto di accumulazione} per $A$ e $\; \displaystyle \exists \, \lim_{P \rightarrow P_0} f(P) = f(P_0)$
	\end{enumerate}

\item $f$ si dice \emph{continua su $A$} se $f$ è continua in ogni punto $P_0 \in A$
\end{enumerate}
\end{definition}

Dalle proprietà sui limiti di funzioni seguono le seguenti proprietà per funzioni continue.

\begin{proposition}
Siano $f,\,g:A \longrightarrow \mathbb{R}$ e supponiamo che $f$ e $g$ siano continue in un punto assegnato $P_0 \in A$. Allora
\begin{enumerate}[labelindent=\parindent,leftmargin=*,label=\textnormal{(\roman*)},start=1]
\item La funzione $\; f+g:A \longrightarrow \mathbb{R}\; $ è continua nel punto $P_0$.
\item La funzione $\; f \cdot g:A \longrightarrow \mathbb{R}\; $ è continua nel punto $P_0$.
\item Se $\; g(P) \neq 0 \quad \forall \, P \in A \;$, la funzione $\; \displaystyle \frac{f}{g}:A \longrightarrow \mathbb{R}\; $ è continua nel punto $P_0$.
\item Se $F:\mathbb{R} \longrightarrow \mathbb{R}$ è continua (in realtà basterebbe richiedere la continuità nel trasformato secondo $f$ di $P_0$, ossia $f(P_0)$),  la funzione $\; \tilde{f} = F \circ f : A \longrightarrow \mathbb{R}\; $ è continua nel punto $P_0$.
\end{enumerate}
\end{proposition}

\begin{obs}[importante]
Consideriamo il seguente esempio:
$$
f:\mathbb{R}^2 \longrightarrow \mathbb{R}, \qquad
f(x,\,y) =
\begin{cases}
\frac{xy}{x^2+y^2} & (x,\,y) \neq (0,\,0) \\
0 & (x,\,y) = (0,\,0)
\end{cases}
$$
Abbiamo visto che $\; \displaystyle \nexists \lim_{P \rightarrow (0,\,0)} f(P)$, pertanto $f$ non è continua nel punto $P_0 = (0,\,0)$.

Ma se noi fissiamo $x \in \mathbb{R}$ e consideriamo $\; \mathbb{R} \ni y \longmapsto f(x,\,y)$, tale funzione è continua (come funzione di una variabile reale). Analogamente accade fissando l'altra variabile.

Quindi non possiamo vedere la continuità ``congelando'' ad una ad una le variabili!
\end{obs}

Introdotta la continuità, a questo punto sorge automaticamente una domanda\ldots

\underline{Problema}: Nozione di ``\emph{derivata/e}'' per una funzione di due variabili.\\
Il problema di trovare una qualche nozione di derivata si traduce in una prima istanza nella definizione delle cosiddette \emph{derivate parziali}.

\begin{definition}[di derivata parziale]
Sia $A \subset \mathbb{R}^2$ aperto\footnote{Se prendo un punto isolato, non faccio variare un bel niente!}, e siano $f:A \longrightarrow \mathbb{R}$, $P_0 = (x_0,\,y_0) \in A$ fissato. Poiché $A$ è aperto, $\exists \, \delta_0 > 0$ tale che:
$$Q \doteqdot \left[ x_0-\delta_0,\,x_0+\delta_0 \right] \times \left[ y_0-\delta_0,\,y_0-\delta_0 \right] \subset A$$
\begin{center}
\def\svgwidth{7cm}
\input{./01_calcolo_in_piu_variabili/figures/rettangolo_partial.pdf_tex}
\end{center}
Fissato $x_0$, possiamo considerare il rapporto incrementale
$$
\left( \left[ -\delta_0,\,\delta_0 \right] \setminus \lbrace 0 \rbrace \right) \ni h \longmapsto
\frac{f(x_0,\,y_0+h)-f(x_0,\,y_0)}{h}
$$
e analogamente, fissato $y_0$,
$$
\left( \left[ -\delta_0,\,\delta_0 \right] \setminus \lbrace 0 \rbrace \right) \ni h \longmapsto
\frac{f(x_0+h,\,y_0)-f(x_0,\,y_0)}{h}
$$
Queste due funzioni sono ben definite. Allora
\begin{enumerate}[labelindent=\parindent,leftmargin=*,label=\textnormal{(\roman*)},start=1]
\item $f$ si dice \emph{derivabile (parzialmente)} rispetto alla direzione $x$ nel punto $P_0$ se
$$
\exists \lim_{h \rightarrow 0} \frac{f(x_0+h,\,y_0)-f(x_0,\,y_0)}{h} \in \mathbb{R}
\overset{def.}{=}
\frac{\partial f}{\partial x}(P_0) = D_1 f(P_0)
$$
\item $f$ si dice \emph{derivabile (parzialmente)} rispetto alla direzione $y$ nel punto $P_0$ se
$$
\exists \lim_{h \rightarrow 0} \frac{f(x_0,\,y_0+h)-f(x_0,\,y_0)}{h} \in \mathbb{R}
\overset{def.}{=}
\frac{\partial f}{\partial y}(P_0) = D_2 f(P_0)
$$
\item Se $\; \displaystyle \exists \, \frac{\partial f}{\partial x}(P_0) \;$ e $\; \displaystyle \exists \, \frac{\partial f}{\partial y}(P_0)$, si dice che esiste il \emph{vettore gradiente} di $f$ nel punto $P_0$ e si denota questo vettore con
$$
\nabla f(P_0) \doteqdot \left( \frac{\partial f}{\partial x}(P_0),\,\frac{\partial f}{\partial y}(P_0) \right) \in \mathbb{R}^2
$$
\end{enumerate}
\end{definition}

\begin{obs}
Se $\; \exists \, \nabla f(P) \;$ in ogni punto di $A$ è definita una funzione vettoriale
$$\nabla f : A \ni \underline{P} \longmapsto \nabla f(P) \doteqdot \left( \frac{\partial f}{\partial x}(P),\,\frac{\partial f}{\partial y}(P) \right) \in \mathbb{R}^2$$
che descrive un cosiddetto \emph{campo vettore}.

Ad esempio, $- \nabla f = \overset{\rightarrow}{G}$, dove $\overset{\rightarrow}{G}$ in fisica è il campo di gravità, è un campo vettore.
\end{obs}

\underline{Problema}: La definizione di vettore gradiente di una funzione in un punto rappresenta una buona nozione di ``derivata''?

\begin{example}
$$
f:\mathbb{R}^2 \longrightarrow \mathbb{R}, \qquad
f(x,\,y) =
\begin{cases}
\frac{xy}{x^2+y^2} & (x,\,y) \neq (0,\,0) \\
0 & (x,\,y) = (0,\,0)
\end{cases}
$$
Si vede (da fare per esercizio) che $\; \exists \, \nabla f(0,\,0) = (0,\,0)$, ma $f$ non è continua in $(0,\,0)$! Noi, invece, vorremmo che la funzione fosse continua per poter fare la ``derivata'', come nel caso di una variabile.
\end{example}

Prendiamo ora in considerazione la nozione di punto tangente al grafico, e introduciamo la definizione di funzione differenziabile in un punto e di differenziale di una funzione.


\section{Differenziale di una funzione}
Ricordiamo cosa accade nel caso $n=1$.

\begin{exer}
Siano $\; A=(a,\,b), \quad x_0 \in (a,\,b), \quad f:(a,\,b) \longrightarrow \mathbb{R}, \quad m \in \mathbb{R}$.
\begin{center}
$f$ è derivabile nel punto $x_0$ e $f'(x_0)=m$
$$\Updownarrow$$
$$\exists \, \lim_{x \rightarrow x_0} \frac{f(x)-\left[m(x-x_0)+f(x_0)\right]}{|x-x_0|} = 0$$
\end{center}
dove (lo ricordiamo), $m(x-x_0)+f(x_0)$ è l'equazione della retta tangente al grafico di $f$ data da $\; y= f(x_0)+f'(x_0)(x-x_0)\;$ e $|x-x_0|=\mathrm{d}(x,\,x_0)$.

Se definiamo l'applicazione lineare $\;L:\mathbb{R}\longrightarrow\mathbb{R}\;$ definita da
$$L(v)=mv, \qquad v \in \mathbb{R}$$
otteniamo che
\begin{center}
$f$ è derivabile nel punto $x_0$ e vale che $f'(x_0)=m$
$$\Updownarrow$$
$\exists \, L:\mathbb{R}\longrightarrow\mathbb{R}\;$ lineare tale che $\; \displaystyle \lim_{x \rightarrow x_0} \frac{f(x)-\left[L(x-x_0)+f(x_0)\right]}{\mathrm{d}(x-x_0)} = 0$
\end{center}
\end{exer}

Vediamo ora cosa succede per $n=2$.

Sia $A \subseteq \mathbb{R}^2$ aperto, $f:A \longrightarrow \mathbb{R}$, $P_0=(x_0,y_0) \in A$.

\begin{center}
\def\svgwidth{12cm}
\input{./01_calcolo_in_piu_variabili/figures/diff_3d.pdf_tex}
\end{center}

\underline{Problema}: Nozione di punto tangente alla superficie $G_f$ (che è una sorta di ``cupola 3D'' che rappresenta il grafico di f) nel punto $(x_0,\,y_0,\,f(x_0,\,y_0))$.

Supponiamo che $\pi$ sia un piano di $\mathbb{R}^3$ non ``verticale'', cioè non parallelo ai piani $xz$ e $zy$ (dove $z=f(x,\,y)$).
Scriviamo innanzitutto l'equazione di $\pi$. La generica retta passante per $\; (x_0,\,y_0)$ nel piano $xy$ è
$$y-y_0=m_1(x-x_0)$$
con $m_1 \in \mathbb{R}$.
Ci serve ora un'altra equazione per il piano $zy$, e per la precisione l'equazione della retta passante per $(f(x_0,\,y_0),y_0)$:
$$z - f(x_0,\,y_0) = m_2(y-y_0)$$
con $m_2 \in \mathbb{R}$. Sommando membro a membro le due equazioni, otteniamo che
$$\pi : z = a_1(x-x_0) + a_2(y-y_0) + f(x_0,\,y_0)$$
con $a_1,\,a_2 \in \mathbb{R}$.
Il nostro obiettivo è determinare $a_1,\,a_2$ tali che $\pi$ sia tangente a $G_f$.

Definiamo l'applicazione $\; L:\mathbb{R}^2 \longrightarrow \mathbb{R}, \quad v=(v_1,\,v_2) \in \mathbb{R}, \quad L(v) \doteqdot a_1v_1 + a_2v_2$. \`E immediato osservare che $L$ è un'applicazione lineare, cioè, per definizione
\begin{enumerate}[labelindent=\parindent,leftmargin=*,label=\textnormal{(Lin\arabic*)},start=1]
\item $L(v+w)=L(v)+L(w) \qquad \forall \, v,\,w \in \mathbb{R}^2$
\item $L(\lambda v) = \lambda L(v) \qquad \forall \, \lambda \in \mathbb{R}, \quad \forall \, v \in \mathbb{R}^2$
\end{enumerate}

\begin{exer}
Mostrare che $\; L:\mathbb{R}^n \longrightarrow \mathbb{R} \;$ è lineare $\quad \Longleftrightarrow \quad \exists \, a=(a_1,\,a_2,\ldots,a_n) \in \mathbb{R}^2 \quad$ tale che $\quad L(v) = a \bullet v \overset{def}{=} \displaystyle \sum_{i=1}^n a_i v_i$.
\end{exer}

Quindi, proseguendo nel nostro ragionamento, possiamo scrivere che
$$\pi : z = a_1(x-x_0) + a_2(y-y_0) + f(x_0,\,y_0) = L(P-P_0) + f(x_0,\,y_0)$$
dove $L$ è una (qualunque) applicazione lineare.

Inoltre, $A$ è aperto, quindi $\; \exists \, \delta_0 > 0 \;$ tale che $\; \mathrm{B}(P_0,\,\delta_0) \subseteq A$. Questo ci consente di dire che è ben definito il seguente ``rapporto incrementale globale'', definito da
$$
\mathrm{B}(P_0,\,\delta_0) \setminus \lbrace P_0 \rbrace \ni P \longmapsto
\frac{f(P)-L(P-P_0)-f(P_0)}{\mathrm{d}(P,\,P_0)}
$$

Dove il numeratore è del tipo ``Funzione''$-$``Piano Tangente''. A questo punto, è naturale dare la seguente definizione.

\begin{definition}
Siano $\; A \subseteq \mathbb{R}^2 \;$ un insieme aperto, $P_0 \in A$ fissato, $f:A \longrightarrow \mathbb{R}$. Si dice che \emph{$f$ è differenziabile nel punto $P_0$} se esiste una applicazione lineare $\; L: \mathbb{R}^2 \longrightarrow \mathbb{R} \;$ tale che

\begin{center}
$\mathrm{(D)}$
\hfill
$\displaystyle
\exists \, \lim_{P \rightarrow P_0} \frac{f(P)-L(P-P_0)-f(P_0)}{\mathrm{d}(P,\,P_0)} =$
\hfill \null \\
\vskip 16pt
\hfill
$\displaystyle
=\lim_{(x,\,y) \rightarrow (0,\,0)} \frac{f(x,\,y) - a_1(x-x_0) - a_2(y-y_0) - f(x_0,\,y_0)}{\sqrt{(x-x_0)^2 - (y-y_0)^2}} =
0
$
\hfill \null
\end{center}

L'applicazione lineare $L$, se esiste, si chiama \emph{differenziale di $f$ nel punto $P_0$} e si denota con
$$df(P_0) \equiv L$$
\end{definition}

\begin{thm}
Sia $A \subseteq \mathbb{R}^2$ aperto, $f:A \longrightarrow \mathbb{R}$, $P_0 \in A$. Supponiamo che $f$ sia differenziabile nel punto $P_0$. Allora
\begin{enumerate}[labelindent=\parindent,leftmargin=*,label=\textnormal{(\roman*)},start=1]
\item $\displaystyle \exists \, \nabla f(P_0) \overset{def}{=} \left( \frac{\partial f}{\partial x}(P_0), \, \frac{\partial f}{\partial y}(P_0) \right) \in \mathbb{R}^2$

Inoltre,
\begin{center}
$\mathrm{(\star)}$
\hfill
$\displaystyle
df(P_0)(v) = \nabla f(P_0) \bullet v = \frac{\partial f}{\partial x}(P_0)v_1 + \frac{\partial f}{\partial y}(P_0)v_2, \qquad \forall \, v=(v_1,\,v_2) \in \mathbb{R}^2$
\hfill \null
\end{center}
Da $(\star)$ segue che $df(P_0)$, se esiste, è univocamente determinata.

\item $f$ è continua in $P_0$.
\end{enumerate}

\end{thm}
\begin{proof}
\begin{enumerate}[labelindent=\parindent,leftmargin=*,label=\textnormal{(\roman*)},start=1]
\item
Sia $e_1=(1,\,0)$ il primo vettore della base canonica di $\mathbb{R}^2$. Scelgo dunque $P=P_0+he_1$ con $|h|<\delta_0$, cioè proseguo partendo da $P_0$ lungo l'asse $x$ beccando $P$. Dalla $(D)$, segue che
	\begin{center}
	$\mathrm{(1)}$
	\hfill
	$\displaystyle \exists \, \lim_{h \rightarrow 0} \frac{f(P_0 + he_1) - f(P_0) - L(he_1)}{|h|} = 0$\hfill\null\\
	\hfill
	$\displaystyle \Longleftrightarrow \exists \, \lim_{h \rightarrow 0} \frac{f(P_0 + he_1) - f(P_0) - hL(e_1)}{h} = 0$\hfill\null\\
	\hfill
	$\displaystyle \Longleftrightarrow \exists \, \lim_{h \rightarrow 0} \frac{f(P_0 + he_1) - f(P_0)}{h} = L(e_1) \in \mathbb{R}$\hfill\null\\
	$\mathrm{(2)}$
	\hfill
	$\displaystyle \overset{def}{\Longleftrightarrow} \exists \, \frac{\partial f}{\partial x}(P_0) = L(e_1)$
	\hfill \null
	\end{center}
Analogamente, restringendosi ai punti del tipo $P=P_0+he_2$, con $e_2=(0,\,1)$, si prova che:
	\begin{center}
	$\mathrm{(3)}$
	\hfill
	$\displaystyle \exists \, \frac{\partial f}{\partial y}(P_0) = L(e_2)$\hfill\null
	\end{center}
Da (2) + (3) segue che $\exists \, \nabla f(P_0)$.

D'altra parte, poiché $(e_1,\,e_2)$ è una base, quindi possiamo scrivere che $v=e_1v_1 + e_2v_2$ per appropriati $v_1,\,v_2 \in \mathbb{R}$. Usando la linearità di $L$, otteniamo che $L(v)= L(e_1)v_1 + L(e_2)v_2$, da cui segue $(\star)$.

\item
Dobbiamo provare che
	\begin{center}
	$\mathrm{(1)}$
	\hfill
	$\displaystyle \exists \, \lim_{P \rightarrow P_0} f(P) = f(P_0)$\hfill\null\\
	\hfill
	$\Updownarrow$
	\hfill\null\\
	$\mathrm{(2)}$
	\hfill
	$\displaystyle \exists \, \lim_{P \rightarrow P_0} (f(P)-f(P_0)) = 0$\hfill\null
	\end{center}
	Se $P \neq P_0$, allora
	$$
	f(P) - f(P_0) =
	\frac{f(P)-f(P_0)-df(P_0)(P-P_0)}{\mathrm{d}(P,\,P_0)}\mathrm{d}(P,\,P_0)+df(P_0)(P-P_0)
	$$
	Per la $\mathrm{(D)}$, sappiamo che $\frac{f(P)-f(P_0)-df(P_0)(P-P_0)}{\mathrm{d}(P,\,P_0)} \underset{_{P \rightarrow P_0}}{\longrightarrow} 0$. D'altra parte, per definizione, $\mathrm{d}(P,\,P_0) \underset{_{P \rightarrow P_0}}{\longrightarrow} 0$. Resta da provare che $df(P_0)(P-P_0) \underset{_{P \rightarrow P_0}}{\longrightarrow} 0$.
	\begin{exer}[Esercizio 5 del Foglio 2]
	Sia $L:\mathbb{R}^n \longrightarrow \mathbb{R}$ \emph{lineare}, allora $L$ è continua su $\mathbb{R}^n$.
	\end{exer}
	Quindi, usando l'esercizio, $df(P_0)(P-P_0) \underset{_{P \rightarrow P_0}}{\longrightarrow}  df(P_0)(0) = 0$.
\end{enumerate}
\end{proof}

\begin{definition}[di piano tangente al grafico]
Data $f:A \longrightarrow \mathbb{R}$, con $A$ insieme aperto di $\mathbb{R}^2$ e $P_0 \in A$, supponiamo che $f$ sia differenziabile nel punto $P_0$. Si chiama \emph{piano tangente a $G_f$\footnote{$G_f$ indica il grafico di $f$.} nel punto $(x_0,\,y_0,\,f(x_0,\,y_0))$} il piano $\pi$ di $\mathbb{R}^3$ di equazione
$$\pi : df(P_0)(P-P_0) + f(P_0), \qquad \forall \, P=(x,\,y) \in \mathbb{R}^2$$
\end{definition}

\subsection{Estensione al caso dimensionale $n \geq 3$}
Siano $A \subseteq \mathbb{R}^n$ aperto, $f: A \longrightarrow \mathbb{R}$, $P^0 = (x_1^0,\ldots,x_n^0) \in A$ fissato. Sia $(e_1,\ldots,e_n)$ la base canonica di $\mathbb{R}^n$.

Essendo $A$ aperto, $\exists \, \delta_0 > 0$ tale che $\mathrm{B}(P^0,\,\delta_0) \subseteq A$. \`E ben definito il rapporto incrementale rispetto alla direzione $i$-esima
$$
(\delta_0,\,\delta_0) \setminus \lbrace P^0 \rbrace \ni h \longmapsto
\frac{f(P^0+he_i)-f(P^0)}{h}
$$

\begin{definition}
Siano $A$, $f$, $P^0$ e $(e_1,\ldots,e_n)$ come sopra.
\begin{enumerate}[labelindent=\parindent,leftmargin=*,label=\textnormal{(\roman*)},start=1]
\item $f$ si dice \emph{derivabile (parzialmente) rispetto alla direzione $x_i$} nel punto $P^0$ se
$$
\exists \, \frac{\partial f}{\partial x_i} (P^0) = \lim_{h \rightarrow 0} \frac{f(P^0+he_i)-f(P^0)}{h} \in \mathbb{R}
$$

\item Se $\displaystyle \exists \, \frac{\partial f}{\partial x_i} (P^0) \; \forall \, i=1,\ldots,n$ si dice che esiste il \emph{vettore gradiente} di $f$ in $P^0$ e si denota con
$$
\nabla f(P^0) = \left( \frac{\partial f}{\partial x_1}(P^0),\ldots,\frac{\partial f}{\partial x_n}(P^0) \right)
$$

\item $f$ si dice differenziabile nel punto $P^0$ se esiste un'applicazione lineare $L:\mathbb{R}^n \rightarrow \mathbb{R}$ per cui vale (D).

L'applicazione $L$ si chiama \emph{differenziale di $F$ in $P^0$} e si denota con $df(P^0):\mathbb{R}^n \rightarrow \mathbb{R}$.
\end{enumerate}
\end{definition}

Si prova ancora che, se $f$ è differenziabile in $P^0$, allora:
\begin{enumerate}[labelindent=\parindent,leftmargin=*,label=\textnormal{(\roman*)},start=1]
\item $\exists \, \nabla f(P^0)$
\item $\displaystyle df(P^0)(v) = \nabla f(P^0) \bullet v = \sum_{i=1}^n \frac{\partial f}{\partial x_i} (P^0)v_i, \qquad v=(v_1,\ldots,v_n) \in \mathbb{R}^n$
\end{enumerate}

\begin{definition}
Si dice \emph{direzione} o \emph{versore} di $\mathbb{R}^n$ un vettore $v \in \mathbb{R}^n$ tale che $||v|| = 1$.
\end{definition}

Siano $A \subseteq \mathbb{R}^n$ aperto, $f: A \longrightarrow \mathbb{R}$, $P^0 = (x_1^0,\ldots,x_n^0) \in A$, $v$ una direzione di $\mathbb{R}^n$.

Essendo $A$ aperto, $\exists \, \delta_0 > 0$ tale che $\mathrm{B}(P_0,\,\delta_0) \subseteq A$. \`E ben definito il rapporto incrementale rispetto alla direzione $v$
$$
(\delta_0,\,\delta_0) \setminus \lbrace P_0 \rbrace \ni h \longmapsto
\frac{f(P_0+hv)-f(P_0)}{h}
$$

\begin{definition}[derivata direzionale]
Si dice che \emph{$f$ è derivabile (parzialmente) rispetto alla direzione $v$} nel punto $P_0$ se
$$
\exists \, \frac{\partial f}{\partial v} (P_0)= \lim_{h \rightarrow 0} \frac{f(P_0+hv)-f(P_0)}{h} \in \mathbb{R}
$$
\end{definition}

\begin{proposition}
Siano $A \subseteq \mathbb{R}^n$ aperto, $f: A \longrightarrow \mathbb{R}$, $P_0 \in A$ fissato. Supponiamo che $f$ sia differenziabile in $P_0$. Allora, per ogni direzione $v \in \mathbb{R}^n$,
$$
\exists \, \frac{\partial f}{\partial v} (P_0) = \nabla f(P_0) \bullet v = df(P_0)(v)
$$
\end{proposition}
\begin{proof}
Scegliamo $P=P_0+hv$. Dalla (D) segue che
$$ \lim_{h \rightarrow 0} \frac{f(P_0+hv)-\overset{\overset{hL(v)}{\shortparallel}}{L(hv)}-f(P_0)}{|h|} = 0 \quad \Longleftrightarrow \quad \exists \, \lim_{h \rightarrow 0} \frac{f(P_0+hv)-f(P_0)}{h} \quad \overset{def}{\Longleftrightarrow}$$
$$\overset{def}{\Longleftrightarrow} \quad df(P_0)(v) = \frac{\partial f}{\partial v} (P_0)$$
Come nel caso bidimensionale, leggendo $v$ sulla base canonica e usando la linearità di $L$ si verifica che
$$
df(P_0)(v) = \nabla f(P_0) \bullet v
$$
\end{proof}

\begin{obs}[$n=1$]
Siano $A \subset \mathbb{R}$ aperto e $f: A \longrightarrow \mathbb{R}$.
\begin{center}
$f$ è differenziabile nel punto $x_0$ $\Longleftrightarrow$ $f$ è derivabile nel punto $x_0$
\end{center}
In altri termini, $df(x_0)(v) = f'(x_0)v$.
\end{obs}

\subsection{Condizioni sufficienti per la differenziabilità}
Abbiamo già visto che, dati $A \subseteq \mathbb{R}$ aperto, $f:A \longrightarrow \mathbb{R}$, $P_0 \in A$ fissato, se $f$ è differenziabile nel punto $P_0$ allora $f$ è continua nel punto $P_0$.

Ma, viceversa, la continuità da sola non basta per garantire la differenziabilità!

Un caso banale (per $n=1$ e $A = \mathbb{R}$) è
$$f(x) = |x|$$
$f$ è continua in $x_0 = 0$, ma $\nexists \, f'(0)$.

Ottenere una caratterizzazione della differenziabilità, in effetti, è difficile\ldots infatti non c'è! Nessuno finora ci è riuscito. Diamo comunque il seguente teorema.

\begin{thm}[della differenziabilità o del differenziale totale]
Siano $A \subseteq \mathbb{R}^n$ aperto, $P_0 \in A$ fissato, $f:A \longrightarrow \mathbb{R}$. Supponiamo che $\exists \, \delta_0 > 0$ tale che 
\begin{center}
\def\svgwidth{6cm}
\input{./01_calcolo_in_piu_variabili/figures/thm_diff.pdf_tex}
\end{center}
$\mathrm{B}(P_0,\,\delta_0) \subseteq A$ e valgano:
\begin{enumerate}[labelindent=\parindent,leftmargin=*,label=\textnormal{(\roman*)},start=1]
\item $\displaystyle \exists \, \frac{\partial f}{\partial x_i} : \mathrm{B}(P_0,\,\delta_0) \longrightarrow \mathbb{R}, \qquad \forall \, i = 1,\,2,\ldots,n$
\item $\displaystyle \frac{\partial f}{\partial x_i} : \mathrm{B}(P_0,\,\delta_0) \longrightarrow \mathbb{R}$ è continua nel punto $P_0$, $\qquad \forall \, i = 1,\,2,\ldots,n$
\end{enumerate}
Allora $f$ è differenziabile nel punto $P_0$.
\end{thm}
\begin{proof}
Dimostriamo il teorema nel caso $n=2$, tenendo presente che la dimostrazione si può estendere in modo analogo anche per $n \geq 3$.

Prendiamo, invece della pallina, un piccolo quadrato inscritto nella circonferenza di raggio $\delta_0$.
\begin{center}
\def\svgwidth{14cm}
\input{./01_calcolo_in_piu_variabili/figures/thm_diff_quad.pdf_tex}
\end{center}
In altri termini, $\exists \, \delta_0 > \delta_1 > 0$ tale che il quadrato $Q=[x_0-\delta_1,\,x_0+\delta_1] \times [y_0-\delta_1,\,y_0+\delta_1] \subseteq \mathrm{B}(P_0,\,\delta_0)$.

Sia ora $P=(x,\,y) \in Q$, con $P \neq P_0$. Fissiamo $y \in [y_0-\delta_1,\,y_0+\delta_1]$ e consideriamo la funzione (di una variabile reale)
$$g:[x_0-\delta_1,\,x_0+\delta_1] \rightarrow \mathbb{R}, \qquad g(x) = \overbrace{f(x,\,y)}^{f \text{ ``congelato''}}$$
Da (i) segue che $g$ è derivabile su $[x_0-\delta_1,\,x_0+\delta_1]$. Possiamo quindi applicare il Teorema del valor medio di Lagrange\footnote{
\cite{Greco2012} Sia $f$ una funzione continua su un intervallo chiuso limitato $[a,\,b]$ di ampiezza non nulla. Se $f$ ha derivata finita in $(a,\,b)$, allora
$$
\underset{{\xi \in (a,\,b)}}{\exists} \, f(b)-f(a) = f'(\xi)(b-a)
$$
} a $g$ su $(x_0,\,x)$, ottenendo che
$$g(x)-g(x_0) = g'(c_x)(x-x_0)$$
per un opportuno $c_x$ compreso tra $x_0$ e $x$.

In termini di $f$, possiamo rileggere l'uguaglianza come
	\begin{center}
	$\mathrm{(1)}$
	\hfill
	$\displaystyle f(x,\,y) - f(x_0,\,y) = \frac{\partial f}{\partial x}((c_x,\,y))(x-x_0) \qquad \forall \, P \in Q$\hfill\null
	\end{center}

Analogamente, consideriamo la funzione (di una variabile reale):
$$h:[y_0-\delta_1,\,y_0+\delta_1] \rightarrow \mathbb{R}, \qquad h(x) = f(x_0,\,y)$$

Di nuovo, otteniamo da (i) che la funzione $h$ è derivabile in $[y_0-\delta_1,\,y_0+\delta_1]$ e possiamo applicare il Teorema del valor medio di Lagrange ottenendo che
$$h(y)-h(y_0) = h'(c_y)(y-y_0)$$
per un opportuno $c_y$ compreso tra $y_0$ e $y$.

In termini di $f$, possiamo rileggere l'uguaglianza come
	\begin{center}
	$\mathrm{(2)}$
	\hfill
	$\displaystyle f(x_0,\,y) - f(x_0,\,y_0) = \frac{\partial f}{\partial y}((x_0,\,c_y))(y-y_0) \qquad \forall \, y \in [y_0-\delta_1,\,y_0+\delta_1]$\hfill\null
	\end{center}

Proviamo ora che da (1)+(2) segue che $f$ è differenziabile in $P_0$ (utilizzando la (ii)), cioè, per definizione, dobbiamo provare che
\begin{center}
$\mathrm{(D)}$
\hfill
$\displaystyle
\exists \, \lim_{P \rightarrow P_0} \frac{f(P)-L(P-P_0)-f(P_0)}{\mathrm{d}(P,\,P_0)} =$
\hfill \null \\
\vskip 16pt
\hfill
$\displaystyle
\exists \, \lim_{P \rightarrow P_0} \frac{f(P)-df(P_0)(P-P_0)-f(P_0)}{\mathrm{d}(P,\,P_0)} =$
\hfill \null \\
\vskip 16pt
\hfill
$\displaystyle
\exists \, \lim_{P \rightarrow P_0} \frac{f(P)-\nabla f(P_0) \bullet (P-P_0)-f(P_0)}{\mathrm{d}(P,\,P_0)} =$
\hfill \null \\
\vskip 16pt
\hfill
$\displaystyle
\exists \, \lim_{(x,\,y) \rightarrow (x_0,\,y_0)} \frac{\displaystyle \left\lvert f(x,\,y) - \frac{\partial f}{\partial x}(P_0)(x-x_0) - \frac{\partial f}{\partial y}(P_0)(y-y_0) - f(x_0,\,y_0) \right\rvert}{\mathrm{d}(P,\,P_0)} = 0$
\hfill \null \\
\end{center}

Osserviamo che da (1) e (2) segue che
\begin{center}
\hfill
$\displaystyle
\frac{\displaystyle \left\lvert (f(x,\,y)-f(x_0,\,y)) + (f(x_0,\,y)-f(x_0,\,y_0)) - \frac{\partial f}{\partial x}(P_0)(x-x_0) - \frac{\partial f}{\partial y}(P_0)(y-y_0) \right\rvert }{\mathrm{d}(P,\,P_0)} =$
\hfill \null \\
\vskip 16pt
\hfill
$\displaystyle
\frac{\displaystyle \left\lvert \overbrace{\left( \frac{\partial f}{\partial x}(c_x,\,y) - \frac{\partial f}{\partial x}(P_0) \right)}^{\doteqdot \, \varphi_1 (x,\,y)} (x-x_0) + \overbrace{\left( \frac{\partial f}{\partial y}(x_0,\,c_y)) - \frac{\partial f}{\partial y}(P_0) \right)}^{\doteqdot \, \varphi_2 (x,\,y)} (y-y_0) \right\rvert }{\mathrm{d}(P,\,P_0)} \overset{\forall \, P \in Q \setminus \left\lbrace P_0 \right\rbrace}{\leq} $
\hfill \null \\
\hfill
$\displaystyle \left\lvert \varphi_1(x,\,y) \right\rvert \overbrace{\left( \frac{|x-x_0|}{\mathrm{d}(P,\,P_0)} \right)}^{\footnotemark} + \left\lvert \varphi_2(x,\,y) \right\rvert \left( \frac{|y-y_0|}{\mathrm{d}(P,\,P_0)} \right) \leq
\displaystyle \left\lvert \varphi_1(x,\,y) \right\rvert + \left\lvert \varphi_2(x,\,y) \right\rvert$
\footnotetext{$\leq 1$ per il Teorema di Pitagora (i cateti sono sempre minori dell'ipotenusa!)}
\hfill \null \\
\end{center}
Osserviamo poi che $\varphi_1$ e $\varphi_2$ vanno a $0$ usando la (ii).

Siamo quindi arrivati alla conclusione che
\begin{center}
$\mathrm{(3)}$
\hfill
$\displaystyle
0 \leq \frac{\displaystyle \left\lvert f(x,\,y)-f(x_0,\,y_0) - \frac{\partial f}{\partial x}(P_0)(x-x_0) - \frac{\partial f}{\partial y}(P_0)(y-y_0) \right\rvert }{\mathrm{d}(P,\,P_0)} \leq$
\hfill \null \\
\vskip 16pt
\hfill
$\displaystyle
\left\lvert \varphi_1(x,\,y) \right\rvert + \left\lvert \varphi_2(x,\,y) \right\rvert \overset{(ii)}{\underset{P \rightarrow 0}{\longrightarrow}} 0 \qquad \forall \, P \in Q \setminus \left\lbrace P_0 \right\rbrace
$
\hfill \null \\
\end{center}
Dalla (3) e dalla (ii), applicando il Teorema dei Carabinieri, segue (D).
\end{proof}


\begin{obs}
Il Teorema del differenziale totale è una condizione \emph{sufficiente}, ma \emph{non necessaria}, per la differenziabilità. Per esempio, si provi che, data
$$
f(x) =
\begin{cases}
\displaystyle x^2sin\left(\frac{1}{x}\right) & x \neq 0 \\
0 & x = 0
\end{cases}
$$
$\exists \, f'(x) \quad \text{per} \quad x \neq 0, \qquad \exists \, f'(0)=0, \qquad \text{ma} \quad \nexists \displaystyle \lim_{x \rightarrow 0} f'(x)$.
\end{obs}

\begin{definition}
Dato $A \subseteq \mathbb{R}^n$,
\begin{enumerate}[labelindent=\parindent,leftmargin=*,label=\textnormal{(\roman*)},start=1]
\item si denota con $C^0(A)$ l'insieme delle funzioni $f : A \longrightarrow \mathbb{R}$ continue in ogni punto di $A$;
\item se $A$ è aperto, si denota con $C^1(A)$ l'insieme delle funzioni $f \in C^0(A)$ e tali che $\displaystyle \frac{\partial f}{\partial x_i} \in C^0(A), \qquad \forall \, i=1,\,2,\ldots,n$.
\end{enumerate}
\end{definition}

\begin{cor}
Se $f \in C^1(A)$, allora $f$ è differenziabile in ogni punto di $A$.
\end{cor}

\begin{obs}
Gli spazi $C^m(A) \; (m=0,\,1)$ sono spazi vettoriali su $\mathbb{R}$. Infatti, la somma di funzioni continue è continua, così come il prodotto di una funzione continua per scalari.

Tuttavia, questi spazi hanno dimensione \emph{infinita}! Non possiamo più lavorare come sugli spazi a dimensione finita. In particolare, crolla il Teorema di Bolzano-Weierstrass.
\end{obs}

\subsection{Derivate (parziali) successive}
$(n=2)$\\
Siano $A \subseteq \mathbb{R}^2$, $f : A \longrightarrow \mathbb{R}$ e supponiamo che $\displaystyle \exists \, \frac{\partial f}{\partial x},\,\frac{\partial f}{\partial y} : A \longrightarrow \mathbb{R}$.
Sia ora $P_0 \in A$ fissato.
Se
$$
\overbrace{
\exists \, \frac{\partial}{\partial x} \left( \frac{\partial f}{\partial x} \right)(P_0) \overset{def}{=} \frac{\partial^2 f}{\partial x^2}(P_0),\qquad
\exists \, \frac{\partial}{\partial y} \left( \frac{\partial f}{\partial y} \right)(P_0) \overset{def}{=} \frac{\partial^2 f}{\partial y^2}(P_0)
}^{\text{derivate seconde pure}}
$$
$$
\overbrace{
\exists \, \frac{\partial}{\partial y} \left( \frac{\partial f}{\partial x} \right)(P_0) \overset{def}{=} \frac{\partial^2 f}{\partial y \partial x}(P_0),\qquad
\exists \, \frac{\partial}{\partial x} \left( \frac{\partial f}{\partial y} \right)(P_0) \overset{def}{=} \frac{\partial^2 f}{\partial x \partial y}(P_0)
}^{\text{derivate seconde miste}}
$$
\vskip 16pt

\underline{Problema}: Se $\displaystyle \exists \, \frac{\partial^2 f}{\partial y \partial x}(P_0) \quad \text{e} \quad \exists \, \frac{\partial^2 f}{\partial x \partial y}(P_0)$, vale che $\displaystyle \frac{\partial^2 f}{\partial y \partial x}(P_0) = \frac{\partial^2 f}{\partial x \partial y}(P_0)$?

\begin{obs}
In generale, può accadere che
\begin{center}
$\mathrm{(\star)}$
\hfill
$\displaystyle \exists \, \frac{\partial^2 f}{\partial y \partial x}(P_0) \neq \exists \, \frac{\partial^2 f}{\partial x \partial y}(P_0)$
\hfill \null \\
\end{center}

Ad esempio (Ex. 11 Foglio 2), si prenda
$$
f: \mathbb{R}^2 \longrightarrow \mathbb{R}, \qquad
f(x,\,y) =
\begin{cases}
\displaystyle \frac{2xy(x^2 - y^2)}{x^2 + y^2} & (x,\,y) \neq (0,\,0) \\
0 & (x,\,y) = (0,\,0)
\end{cases}
$$
Si provi che vale $(\star)$ nel punto $P_0 = (0,\,0)$
\end{obs}

\begin{thm}
Siano $A \subset \mathbb{R}^2$ aperto, $f : A \longrightarrow \mathbb{R}$, $P_0 \in A$. Supponiamo che
\begin{enumerate}[labelindent=\parindent,leftmargin=*,label=\textnormal{(\roman*)},start=1]
\item $\displaystyle \exists \, \frac{\partial^2 f}{\partial x \partial y},\,\frac{\partial^2 f}{\partial y \partial x} : A \longrightarrow \mathbb{R}$.
\item $\displaystyle \frac{\partial^2 f}{\partial x \partial y},\,\frac{\partial^2 f}{\partial y \partial x}$ sono continue nel punto $P_0$.
\end{enumerate}
Allora
$$ \exists \, \frac{\partial^2 f}{\partial y \partial x}(P_0) = \exists \, \frac{\partial^2 f}{\partial x \partial y}(P_0)$$
\end{thm}
\begin{proof}
(Si veda \cite{Giusti2003}).
\end{proof}

Osserviamo che il caso si estende (banalmente) anche al caso $n \geq 3$.

\begin{definition}
Sia $A \subseteq \mathbb{R}^n$ aperto. Fissato un intero $m \geq 1$, si denota con $C^m(A)$ l'insieme delle funzioni $f : A \longrightarrow \mathbb{R}$ tali che $f$ è continua e, comunque fissato $1 \leq k \leq m$,
$$
\exists \, \frac{\partial^k f}{\partial x_{i_1} \cdot \ldots \cdot \partial x_{i_k}} : A \longrightarrow \mathbb{R} \qquad \text{continua} \qquad \forall \, i_1,\,\ldots,\,i_k \in \lbrace 1,\,\ldots,\,n \rbrace
$$
\end{definition}

\begin{obs}[importante]
Data $f \in C^m(A)$, dato $1 \leq k \leq m$, e data
$$
\frac{\partial^k f}{\partial x_{i_1} \cdot \ldots \cdot \partial x_{i_k}}(x)
\overset{\text{Teorema di Schwarz}}{=}
\frac{\partial^k f}{\partial x_1^{\alpha_1} \cdot \ldots \cdot \partial x_n^{\alpha_n}}(x)
$$
dove $\alpha_i \in \mathbb{N}$ (eventualmente non nulli), con $\alpha_1+\ldots+\alpha_n=k$.
\end{obs}

\begin{definition}
Siano $A \subseteq \mathbb{R}^n$ aperto, $\mathbb{N} \ni m \geq 0$. Allora, $C^m(A)$ denota l'insieme delle funzioni $f: A \longrightarrow \mathbb{R}$ per cui $\; \exists \, D^\alpha f : A \longrightarrow \mathbb{R} \;$ continua $\; \forall \, \alpha = (\alpha_1,\,\ldots,\,\alpha_n) \in \mathbb{N}^n, \quad |\alpha| \doteqdot \alpha_1 + \ldots + \alpha_n \leq m$, dove
$$
D^\alpha f(x) \overset{def}{=} \frac{\partial^k f}{\partial x_1^{\alpha_1} \cdot \ldots \cdot \partial x_n^{\alpha_n}}(x), \qquad x \in A
$$
\end{definition}




\section{Regola di derivazione composta (RDC)}

\subsection{Regola di composizione di una funzione composta}
Vediamo un caso particolare.

Siano $g:[a,\,b] \longrightarrow \mathbb{R}^2$, $g(t) = (x(t),\,y(t))$ derivabile $\forall \, t$, cioè $\exists \, x'(t), \, y'(t) \quad \forall \, t \in [a,\,b]$, ed $f: \mathbb{R}^2 \longrightarrow \mathbb{R}$, $f \in C^1(\mathbb{R}^2)$.

Consideriamo la funzione $h = f \circ g : [a,\,b] \longrightarrow \mathbb{R}$.

\underline{Problema}: $\exists \, h'(t) \overset{???}{=} \nabla f(g(t)) \bullet g'(t)$

\begin{definition}
Sia chiama \emph{curva in $\mathbb{R}^n$} una funzione $\; g : [a,\,b] \subseteq \mathbb{R} \longrightarrow \mathbb{R}^n$.

$\Gamma = g([a,\,b])$ si chiama \emph{supporto} della curva.
\end{definition}

\begin{proposition}
Sia $g:[a,\,b] \longrightarrow \mathbb{R}^n$ derivabile nel punto $t_0 \in [a,\,b]$, cioè, se $g(t) = (x_1(t),\ldots,x_n(t)) \quad \exists \, g'(t_0) = (x'_1(t_0),\ldots,x'_n(t_0)) \in \mathbb{R}^n$.

Sia $A \subseteq \mathbb{R}^n$ aperto e sia $f : A \longrightarrow \mathbb{R}$ con $f \in C^1(A)$ (in effetti, basterebbe che $f$ fosse differenziabile nel punto $P_0 = g(t_0)$).

Supponiamo inoltre che $\; \Gamma = g([a,\,b]) \subseteq A$.

Allora la funzione composta $h = f \circ g : [a,\,b] \longrightarrow \mathbb{R}$ è derivabile nel punto $t_0$ e
$$
h'(t) = \nabla f(g(t_0)) \bullet g'(t_0)
$$
\end{proposition}
\begin{proof}
Sia $t_0 \in (a,\,b)$. Per definizione, $h(t) \doteqdot f(g(t)), \quad t \in [a,\,b]$. Consideriamo, se $t \in [a,\,b] \setminus \lbrace t_0 \rbrace$, il rapporto:
\begin{center}
$\mathrm{(1)}$
\hfill
$\displaystyle \frac{h(t)-h(t_0)}{t-t_0} = \frac{f(g(t))-f(g(t_0))}{t-t_0}$
\hfill \null \\
\end{center}
Per ipotesi, $f$ è differenziabile nel punto $P_0 \doteqdot g(t_0)$, per cui possiamo scrivere
\begin{center}
$\mathrm{(2)}$
\hfill
$\displaystyle f(P) = f(P_0) +df(P_0)(P-P_0) + \sigma(P) = 0, \qquad \forall \, P \in A$
\hfill \null \\
\end{center}
dove
\begin{center}
$\mathrm{(3)}$
\hfill
$\displaystyle \lim_{P \rightarrow P_0} \frac{\sigma(P)}{||P-P_0||} = 0$
\hfill \null \\
\end{center}
Quindi, prendendo $P = g(t)$, otteniamo:
$$\mathrm{(1)} \overset{\mathrm{(3)}}{=} \frac{df(P_0)(g(t)-g(t_0)) + \sigma(g(t))}{t-t_0} = \frac{\nabla f(P_0) \bullet (g(t)-g(t_0)) + \sigma(g(t))}{t-t_0}  = $$
$$ = \nabla f(P_0) \bullet \frac{(g(t)-g(t_0))}{t-t_0} + \frac{\sigma(g(t))}{t-t_0}$$

Se mostriamo che, passando a limite, il secondo addendo di questa somma va a $0$, completiamo la dimostrazione. Pertanto, definiamo la funzione
$$
\sigma^*(P) =
\begin{cases}
\displaystyle \frac{\sigma(P)}{||P-P_0||} & P \neq P_0 \\
0 & P = P_0
\end{cases}
$$
che è continua per la (3). Ricordando che $P = g(t)$, $P_0 = g(t_0)$ e che poiché $g$ è derivabile è anche continua, consentendoci di passare dal limite per $t \rightarrow t_0$ a quello per $P \rightarrow P_0$, possiamo scrivere che
$$\lim_{t \rightarrow t_0} \frac{\sigma(g(t))}{t-t_0} = \lim_{t \rightarrow t_0} \sigma^*(g(t)) \frac{||P-P_0||}{t-t_0} = 0$$
A questo punto, otteniamo che
$$
\lim_{t \rightarrow t_0} \nabla f(P_0) \bullet \frac{(g(t)-g(t_0))}{t-t_0} + \frac{\sigma(g(t))}{t-t_0} =
\nabla f(g(t_0)) \bullet g'(t_0)
$$
\end{proof}

Una prima applicazione della regola di derivazione di una funzione composta è il seguente teorema.

\begin{thm}[del valor medio]
Sia $A \subseteq \mathbb{R}^2 aperto$ e sia $f \in C^1(A)$. Supponiamo che $A$ sia \emph{convesso}, cioé, per definizione, per ogni coppia di punti $P_1,\,P_2 \in A$, il segmento
$$\overline{P_1P_2} \doteqdot \lbrace tP_1+(1-t)P_2 : 0 \leq t \leq 1 \rbrace \subset A$$

\begin{center}
\def\svgwidth{8cm}
\input{./01_calcolo_in_piu_variabili/figures/thm_mean.pdf_tex}
\end{center}

Allora, $\forall \, P_1,\,P_2 \in A, \quad f(P_2)-f(P_1) = \nabla f(P^*)(P_2-P_1)$ per un opportuno punto $P^* \in \overline{P_1P_2}$.
\end{thm}
\begin{proof}
Definiamo la curva $g:[0,\,1] \rightarrow \mathbb{R}^n$ tale che
$$
g(t) = tP_2 + (1-t)P_1 = P_1 + t(P_2-P_1), \qquad 0 \leq t \leq 1
$$
Abbiamo che $g$ è derivabile su $[0,\,1]$ e
$$
g'(t) = P_2-P_1 \qquad \forall \, t \in [0,\,1]
$$
Consideriamo la funzione composta
$$
h = f \circ g : [0,\,1] \rightarrow \mathbb{R} \qquad \text{tale che} \qquad h(t) = f(g(t))
$$
Per la proposizione precedente, $h$ è derivabile su $[0,\,1]$ e
$$
h'(t) = \nabla f(g(t)) \bullet g'(t) = \nabla f(g(t)) \bullet (P_2 - P_1) \qquad \forall \, t \in [0,\,1]
$$
Applichiamo il Teorema del valor medio di Lagrange in una variabile alla funzione $h$ ottenendo:
$$
h(1)-h(0) = h'(t^*)(1-0) = h'(t^*)
$$
per un opportuno $0 \leq t^* \leq 1$. Rileggendo in termini di $f$ e di $g$, possiamo scrivere (ricordando che $g(t^*) = P^* \quad \text{e che} \quad g(1)=P_2, \quad g(0) = P_1 \quad \Longrightarrow \quad h(1) = f(P_2), \quad h(0) = f(P_1)$):
$$
f(P_2) - f(P_1) = \nabla f(P^*) \bullet (P_2-P_1)
$$
\end{proof}

\begin{cor}
Sia $A \subseteq \mathbb{R}^2$ \emph{aperto} e \emph{convesso}, e sia $f \in C^1(A)$. Supponiamo che $\nabla f(P) = \underline{0} = (0,\ldots,0) \quad \forall \, P \in A$.

Allora, fissato un $P_0 \in A$, $f(P)=f(P_0) \quad \forall \, P \in A$.
\end{cor}
\begin{proof}
Fissiamo $P_0 \in A$. Preso $P \in A$, per il teorema del valor medio $\exists \, P^* \in \overline{P_0P}$ tale che
$$
f(P) - f(P_0) = \nabla f(P^*) \bullet (P-P_0)
$$
Ma $\nabla f(P) = 0 \quad \forall \, P$ per ipotesi, quindi sarà nullo anche in $P_0$. L'uguaglianza precedente diventa perciò
$$
f(P) - f(P_0) = 0 \quad \Longrightarrow \quad f(P) = f(P_0) 
$$ 
\end{proof}

\begin{definition}
Sia $A \subseteq \mathbb{R}^n$ aperto.
\begin{enumerate}[labelindent=\parindent,leftmargin=*,label=\textnormal{(\roman*)},start=1]
\item $A$ si dice \emph{sconnesso} se esistono due insiemi $B,\,C$ aperti, disgiunti e non vuoti tali che $A = B \sqcup C$.
\item $A$ si dice \emph{connesso} se non è sconnesso.
\end{enumerate}
\end{definition}

\begin{cor}
Sia $A \subseteq \mathbb{R}^2$ \emph{aperto} e \emph{connesso}, e sia $f \in C^1(A)$. Supponiamo che $\nabla f(P) = \underline{0} = (0,\ldots,0) \quad \forall \, P \in A$.

Allora, fissato un $P_0 \in A$, $f(P)=f(P_0) \quad \forall \, P \in A$.
\end{cor}
\begin{proof}
(Si veda \cite{Giusti2003}).
\end{proof}

\begin{exer}
Sia $A \subseteq \mathbb{R}^n$ aperto e convesso. Provare che $A$ è anche connesso (utilizzando la definizione di connessione).
\end{exer}
\begin{proof}
\underline{(Idea)}

La dimostrazione procede per assurdo. Supponiamo che $A = B \sqcup C$, con $B,\,C$ aperti, disgiunti e non vuoti. Se $A$ è convesso, c'è l'assurdo che devo ``bucare'' l'interfaccia che crea l'unione disgiunta per poter unire un punto di $B$ e un punto di $C$, ossia per unire due particolare punti di $A$!

\begin{center}
\def\svgwidth{8cm}
\input{./01_calcolo_in_piu_variabili/figures/interfaccia.pdf_tex}
\end{center}

Tuttavia, l'idea intuitiva di dover ``bucare'' l'interfaccia non basta. Chi mi assicura che devo bucarla?

Per rispondere, bisogna far uso dell'Assioma di Completezza. Infatti, in $\mathbb{Q}$ tale interfaccia non si bucherebbe.
\end{proof}


\underline{Problema}: Siano $g: \mathbb{R}^n \longrightarrow \mathbb{R}^m$, $f: \mathbb{R}^m \longrightarrow \mathbb{R}^k$ di ``classe $C^1$''. Definendo $h = f \circ g : \mathbb{R}^n \longrightarrow \mathbb{R}^k$, $\; \exists\,  \nabla h(x_1,\ldots,x_n)$?


\subsection{Funzioni continue e differenziabili da $\mathbb{R}^n$ a $\mathbb{R}^m$}
Ricordiamo innanzitutto che $(\mathbb{R}^n,\,||\cdot||_n)$ e $(\mathbb{R}^m,\,||\cdot||_m)$ sono due spazi metrici (e in particolare due spazi topologici). Inoltre, $\mathbb{R}^n$ e $\mathbb{R}^m$ possiedono una struttura di spazio vettoriale lineare su $\mathbb{R}$.

\begin{definition}
Siano $A \subseteq \mathbb{R}^n$, $f : A \longrightarrow \mathbb{R}^m$, $P_0 \in A$ fissato.
\begin{enumerate}[labelindent=\parindent,leftmargin=*,label=\textnormal{(\roman*)},start=1]
\item Se $P_0$ è di accumulazione, allora $f$ si dice \emph{continua nel punto $P_0$} se
$$\exists \, \lim_{P \rightarrow P_0} ||f(P) - f(P_0)||_m = 0$$
Una funzione $f$ si dice continua sull'insieme $A$ se $f$ è continua in ogni punto $P_0 \in A$.
\item Supponiamo che $A \subseteq \mathbb{R}^n$. Si dice che $f$ è differenziabile nel punto $P_0$ se esiste un'applicazione lineare $L=df(P_0) : \mathbb{R}^n \longrightarrow \mathbb{R}^m$ tale che valga
\begin{center}
$\mathrm{(D)}$
\hfill
$\displaystyle \lim_{P \rightarrow P_0} \frac{||f(P)-f(P_0)-L(P-P_0)||_m}{||P-P_0||_n} = 0$
\hfill \null \\
\end{center}
Si dice che $f$ è differenziabile sull'insieme $A$ se $f$ è differenziabile in ogni punto $P_0 \in A$.
\end{enumerate}
\end{definition}

Supponiamo ora di fissare la base canonica di $\mathbb{R}^m \quad (e_1,\ldots,e_m)$. Allora, data $f : \mathbb{R}^n \supseteq A \longrightarrow \mathbb{R}^m$, quest'ultima si può rappresentare come
$$
\underset{\overset{\rotatebox{270}{$\in$}}{\mathbb{R}^m}}{f(P)} = \underset{\overset{\rotatebox{270}{$\in$}}{\mathbb{R}}}{f_1(P)}e_1 + f_2(P)e_2 + \ldots + f_m(P)e_m \equiv (f_1(P),\ldots,f_m(P))
$$
dove $f_i : A \longrightarrow \mathbb{R} \quad (i=1,\ldots,m)$.

\begin{proposition}
Sia $A \subseteq \mathbb{R}^n$, $f: A \longrightarrow \mathbb{R}^m$, $f=(f_1,\ldots,f_m)$. Allora $f$ è continua in ogni punto $P_0 \quad \Longleftrightarrow \quad f_i : A \longrightarrow \mathbb{R}$ è continua nel punto $P_0$ per ogni $i=(1,\ldots,m)$.
\end{proposition}
\begin{proof}
Ricordiamo innanzitutto che (Ex. 1, Foglio 3), per ogni vettore $v \in \mathbb{R}^m$, vale
\begin{center}
$\mathrm{(\star)}$
\hfill
$\displaystyle \frac{1}{\sqrt{m}} \sum_{i=1}^m|v_i| \leq ||v||_m \leq \sum_{i=1}^m|v_i|$
\hfill \null \\
\end{center}
Da $(\star)$ segue che, prendendo $v=f(P)-f(P_0)$,
$$
\frac{1}{\sqrt{m}} \sum_{i=1}^m|f_i(P)-f_i(P_0)| \leq ||f(P)-f(P_0)||_m \leq \sum_{i=1}^m|f_i(P)-f_i(P_0)|
$$
Ora, se un vettore va a $0$, ciascuna componente va a $0$, e viceversa. Quindi, usando opportunamente in base al verso da dimostrare la prima o la seconda disuguaglianza, segue banalmente la tesi (applicando la definizione di continuità!).
\end{proof}

Ricordiamo che, data $L: \mathbb{R}^n \longrightarrow \mathbb{R}^m$ \emph{lineare}, possiamo rappresentarla come
$$
L(v) = (L_1(v),\ldots,L_m(v)), \qquad \forall \, v \in \mathbb{R}^n
$$
dove $L_i: \mathbb{R}^n \longrightarrow \mathbb{R}$ è \emph{lineare} $\forall \, i=1,\ldots,m$.

\begin{proposition}
Sia $A \subseteq \mathbb{R}^n$ aperto, $f: A \longrightarrow \mathbb{R}^m$, $f=(f_1,\ldots,f_m)$. Allora $f$ è differenziabile in un punto $P_0 \quad \Longleftrightarrow \quad f_i : A \longrightarrow \mathbb{R}$ è differenziabile nel punto $P_0 \quad \forall \, i=1,\ldots,m$.

Inoltre:
\begin{center}
$\mathrm{(1)}$
\hfill
$\displaystyle df(P_0)(v) = \Big( df_1(P_0)(v),\,df_2(P_0)(v),\ldots,df_m(P_0)(v) \Big), \qquad \forall \, v \in \mathbb{R}^n$
\hfill \null \\
\end{center}
\end{proposition}
\begin{proof}
Dalla $(\star)$ della proposizione precedente segue che:
$$
\frac{1}{\sqrt{m}} \sum_{i=1}^m \frac{|f_i(P)-f_i(P_0)-L_i(P-P_0)|}{||P-P_0||_n} \leq \frac{||f(P)-f(P_0)-L(P-P_0)||_m}{||P-P_0||} \leq
$$
$$
\leq \sum_{i=1}^m \frac{|f_i(P)-f_i(P_0)-L_i(P-P_0)|}{||P-P_0||_n}
$$
$(\Leftarrow)$\\
$\forall\, i=1,\ldots,m$
$$
\exists \, \lim_{P \rightarrow P_0} \frac{|f_i(P)-f_i(P_0)-L_i(P-P_0)|}{||P-P_0||_n} = 0 \quad \Longrightarrow 
$$
$\Longrightarrow \quad f$ è differenziabile nel punto $P_0$ e $df_i(P_0) = L_i, \quad \forall \, i=1,\ldots,m$.\\
$(\Rightarrow)$\\
Allo stesso modo, procedendo al contrario.
\end{proof}

\begin{cor}
Sia $A \subseteq \mathbb{R}^n$ aperto, $P_0 \in A$ fissato, $f : A \longrightarrow \mathbb{R}^m$. Supponiamo inoltre che $f$ sia differenziabile nel punto $P_0$. Allora
\begin{enumerate}[labelindent=\parindent,leftmargin=*,label=\textnormal{(\roman*)},start=1]
\item $\displaystyle \exists \, \frac{\partial f_i}{\partial x_j} (P_0) \qquad \forall \, i=1,\ldots,m$
\item $df(P_0)(v) = \Big(J(f)\Big)(P_0) \cdot v$, con $v=(v_1,\ldots,v_n)^T$ un vettore colonna,
$$
J(f)(P_0) = \left[ J_{ij} \right]_{m \times n} \qquad \text{dove} \qquad J_{ij} = \frac{\partial f_i}{\partial x_j} (P_0)
$$
e $J(f)(P_0)$ si chiama \emph{matrice jacobiana di $f$ nel punto $P_0$}.
\end{enumerate}
\end{cor}
\begin{proof}
Dalla proposizione precedente, segue che $f_i$ è dfferenziabile e 
\begin{center}
$\mathrm{(2)}$
\hfill
$\displaystyle df_i(P_0)(v) = \nabla f_i(P_0) \bullet v, \quad \forall \, v \in \mathbb{R}^n$
\hfill \null \\
\end{center}
In particolare, dalla $(2)$ segue la $(i)$: fissato $i$, esistono le derivate parziali per ogni $j$.

D'altra parte, da $(1)+(2)$ segue che
$$
J(f)(P_0) = \left(
\begin{array}{c}
\Rigaln{\nabla f_1 (P_0)}\\
\Rigaln{\nabla f_2 (P_0)}\\
\vdots\\
\Rigaln{\nabla f_m (P_0)}\\
\end{array} \right)
^{\footnotemark}
$$
\footnotetext{Le righe orizzontali indicano informalmente dei vettori riga.}
e
$$
df(P_0)(v) = J(f)(P_0) \cdot v, \qquad \forall \, v \in \mathbb{R}^n
$$
\end{proof}

\begin{example}
$m=1$, $f : A \subseteq \mathbb{R}^n \longrightarrow \mathbb{R}^{\xcancel{1}}$. Supponiamo che $f$ sia differenziabile nel punto $P_0$. Allora
$$
df(P_0)(v) = J(f)(P_0) \cdot v
$$
dove
$$
J(f)(P_0) \equiv \nabla f(P_0) \doteqdot \left( \frac{\partial f}{\partial x_1}(P_0), \ldots ,\frac{\partial f}{\partial x_n}(P_0) \right)_{1 \times n}
$$
\end{example}

Poniamoci ora nella seguente situazione:
\begin{center}
\def\svgwidth{14cm}
\input{./01_calcolo_in_piu_variabili/figures/scatole.pdf_tex}
\end{center}

\begin{thm}
Siano $A \subseteq \mathbb{R}^n$ aperto, $x_0 \in A$ fissato, $g : A \longrightarrow \mathbb{R}^m$, $y_0 \doteqdot g(x_0)$, $B \subseteq \mathbb{R}^m$ aperto e sia $f : B \longrightarrow \mathbb{R}^k$.

Supponiamo che:
\begin{enumerate}[labelindent=\parindent,leftmargin=*,label=\textnormal{(\roman*)},start=1]
\item $g(A) \subseteq B$
\item $g$ sia differenziabile nel punto $x_0$
\item $f$ sia differenziabile nel punto $y = g(x_0)$
\end{enumerate}
Allora l'applicazione composta $h = f \circ g : A \longrightarrow \mathbb{R}^k$ è differenziabile nel punto $x_0$ e
\begin{center}
$\underset{\text{\tiny rispetto ai differenziali}}{\mathrm{(RDC)}}$
\hfill
$\displaystyle dh(x_0) = df(g(x_0)) \circ dg(x_0)$
\hfill \null \\
\end{center}
\end{thm}

\begin{obs}[importante]
Operativamente, risulterebbe più semplice avere la RDC rispetto alle jacobiane. Tale formula si può in effetti ottenere, e risulta:
\begin{center}
$\underset{\text{\tiny rispetto alle jacobiane}}{\mathrm{(RDC)}}$
\hfill
$\displaystyle \underbrace{J(h)(x_0)}_{k \times n} = \underbrace{J(f)(g(x_0))}_{k \times \cancel{m}} \circ \underbrace{J(g)(x_0)}_{\cancel{m} \times n}$
\hfill \null \\
\end{center}
se $h=h(x)=(h_1(x),\ldots,h_k(x))$, $y=g(x)$, $f=f(y)$. Quindi,
$$
\frac{\partial h_i}{\partial x_j}(x_0) = \sum_{h=1}^{m} \frac{\partial f_i}{\partial y_h}(g(x_0)) \cdot \frac{\partial g_h}{\partial x_j}(x_0), \qquad \forall \, \underset{\text{\normalsize $j=1,\ldots,n$}}{i=1,\ldots,k}
$$
\end{obs}

\begin{example}[\underline{Applicazione importante: derivazione in coordinate polari (nel piano)}]

Sia $g : (0,\,+\infty) \times (0,\,2\pi) \longrightarrow \mathbb{R}^2$ tale che:
$$
g(\rho,\,\vartheta) = \left( \rho\cos(\vartheta) ,\, \rho\sin(\vartheta) \right)
$$
Consideriamo l'applicazione composta
$$
u(\rho,\,\vartheta) = (f \circ g)((\rho,\,\vartheta)) = f\left( \rho\cos(\vartheta) ,\, \rho\sin(\vartheta) \right)
$$
\underline{Problema}:
$$
\frac{\partial u}{\partial \rho} = ?? \qquad \frac{\partial u}{\partial \vartheta} = ??
$$
\end{example}

Diamo ora la dimostrazione della RDC.

\begin{proof}
(RDC rispetto ai differenziali)\\
Dobbiamo provare che, definita la funzione lineare $L : \mathbb{R}^n \longrightarrow \mathbb{R}^k$ tale che
$$
L(v) = \Big( df(g(x_0)) \circ dg(x_0) \Big) (v) = \Big( df(g(x_0)) \Big) \big( dg(x_0)(v) \big) = $$
$$
= df(y_0)(dg(x_0)(v)), \qquad v \in \mathbb{R}^n
$$
vale:
\begin{center}
$\mathrm{(D)}$
\hfill
$\displaystyle \exists \lim_{P \rightarrow P_0} \frac{||h(P)-h(P_0)-L(P-P_0)||_k}{||P-P_0||_n} = 0 $
\hfill \null \\
\end{center}
Osserviamo che, da $(ii)$ e $(iii)$, segue che:
\begin{enumerate}[labelindent=\parindent,leftmargin=*,label=\textnormal{(\arabic*)},start=1]
\item $g(x) = g(x_0) + dg(x_0)(x-x_0) + \sigma(x), \qquad \forall \, x \in A$
\item $f(y) = f(y_0) + df(y_0)(y-y_0) + \rho(y), \qquad \forall \, y \in B$
\end{enumerate}
con $\sigma : A \longrightarrow \mathbb{R}^m$ e $\rho : B \longrightarrow \mathbb{R}^k$ verificanti:
\begin{enumerate}[labelindent=\parindent,leftmargin=*,label=\textnormal{(\arabic*)},start=3]
\item $\displaystyle \lim_{x \rightarrow x_0} \frac{||\sigma(x)||_m}{||x-x_0||_n} = 0$
\item $\displaystyle \lim_{y \rightarrow y_0} \frac{||\rho(y)||_k}{||x-x_0||_m} = 0$
\end{enumerate}
Volendo riscrivere $(D)$ in termini di $f$ e $g$, osserviamo che:
\begin{enumerate}[labelindent=\parindent,leftmargin=*,label=\textnormal{(\arabic*)},start=5]
\item $\underline{\underline{h(x)-h(x_0)}} =$\\$
f(g(x)) - f(g(x_0)) 
\overset{\text{\tiny $(y = g(x), \; y_0 = g(x_0)) + (2)$}}{=}$\\$
df(y_0)(g(x)-g(x_0)) + \rho(g(x)) \overset{(1)}{=}$\\$
df(y_0)\Big( dg(x_0)(x-x_0) + \sigma(x) \Big) + \rho(g(x)) \overset{\overset{\text{il differenziale}}{\text{\tiny è lineare!}}}{=}$\\$
\underbrace{df(y_0)\Big( dg(x_0)(x-x_0) \Big)}_{L(x-x_0)} + df(y_0)(\sigma(x)) + \rho(g(x)) =$\\$
\underline{\underline{L(x-x_0) + \overbrace{df(y_0)(\sigma(x)) + \rho(g(x))}^{\tau(x)}}}
$
\end{enumerate}
Dalla $(5)$ segue che per provare $(D)$, basta mostrare che:
\begin{enumerate}[labelindent=\parindent,leftmargin=*,label=\textnormal{(\arabic*)},start=6]
\item $\displaystyle \exists \, \lim_{x \rightarrow x_0} \frac{||\tau(x)||_k}{||x-x_0||_n} = 0 \qquad \qquad (P \equiv x,\; P_0 \equiv x_0)$
\end{enumerate}
Osserviamo che:
$$
\frac{||\tau(x)||_k}{||x-x_0||_n} \leq \frac{||df(y_0)(\sigma(x))||_k}{||x-x_0||_n} + \frac{||\rho(g(x))||_k}{||x-x_0||_n}  \qquad \forall \, x \in A \setminus \lbrace x_0 \rbrace
$$
e:
$$
\frac{||df(y_0)(\sigma(x))||_k}{||x-x_0||_n} \overset{\text{linearità!}}{=}
\left\lvert \left\lvert df(y_0) \left( \frac{\sigma(x)}{||x-x_0||_n} \right) \right\lvert \right\lvert_k
$$
Ma il differenziale è un'applicazione lineare! Quindi il suo valore su un vettore è sempre scrivibile come una certa matrice (per la precisione, la jacobiana) per il vettore stesso. Inoltre (Ex. 2, Foglio 3), se $A=(a_{ij})_{m \times n}$ e $v \in \mathbb{R}^n$, vale:
\begin{center}
$\mathrm{(\star)}$
\hfill
$\displaystyle ||Av||_{\mathbb{R}^m} \leq ||A|| \cdot ||v||_n$
\hfill \null \\
\end{center}
dove $\displaystyle ||A|| \overset{def}{=} \sqrt{\sum_{i=1}^m \sum_{j=1}^n a_{ij}^2}$.
Da $(3)+(\star)$ abbiamo quindi che
$$
\left\lvert \left\lvert df(y_0) \left( \frac{\sigma(x)}{||x-x_0||_n} \right) \right\lvert \right\lvert_k \underset{x \rightarrow x_0}{\longrightarrow} 0
$$
Per concludere e provare la $(6)$, basta ora mostrare che:
\begin{enumerate}[labelindent=\parindent,leftmargin=*,label=\textnormal{(\arabic*)},start=7]
\item $\displaystyle \exists \, \lim_{x \rightarrow x_0} \frac{||\rho(g(x))||_k}{||x-x_0||_n} = 0$
\end{enumerate}
Definiamo:
$$
\rho^* : B \longrightarrow \mathbb{R}^k, \qquad
\rho^*(y) \doteqdot
\begin{cases}
\displaystyle \frac{\rho(y)}{||y-y_0||_m} & y \neq y_0 \\
\underline{0}_{\mathbb{R}^k} & y = y_0
\end{cases}
$$
Da $(4)$, segue che $\displaystyle \exists \lim_{y \rightarrow y_0} ||\rho(y)||_k = 0$. Osserviamo ora che
\begin{enumerate}[labelindent=\parindent,leftmargin=*,label=\textnormal{(\arabic*)},start=8]
\item $\displaystyle \frac{||\rho(g(x))||_k}{||x-x_0||_n} =
||\rho^*(g(x))||_k \cdot \frac{||g(x)-g(x_0)||_m}{||x-x_0||_n} \qquad \forall \, x \in A \setminus \lbrace x_0 \rbrace
$
\end{enumerate}
e $$ \frac{||g(x)-g(x_0)||_m}{||x-x_0||_n} \underset{x \rightarrow x_0}{\longrightarrow} 0$$
Per quanto riguarda il secondo fattore, usando la differenziabilità di $g$ in $x_0$ è facile osservare che (Ex. 3, Foglio 3) $\exists \, c > 0, \; R_0 > 0$ tale che:
\begin{enumerate}[labelindent=\parindent,leftmargin=*,label=\textnormal{(\arabic*)},start=8]
\item $\displaystyle \frac{||g(x)-g(x_0)||_m}{||x-x_0||_n} \leq c \qquad \forall \, x \in \mathrm{B}_n (x_0,\,R_0) \setminus \lbrace x_0 \rbrace$
\end{enumerate}
Da $(8)+(9)$, segue la $(6)$.
\end{proof}



\section{Formula di Taylor}
\underline{Problema}: Sia $f \in C^m(\mathrm{B}(x^0,\,r_0))$, dove $x^0=(x^0_1,\ldots,x^0_n) \in \mathbb{R}^n, \; r_0 > 0, \; m \in \mathbb{N}_{\geq 1}$.
Come si fa ad ``approssimare $f$'' con un polinomio $p=p(x_1,\ldots,x_n)$ nel modo ``migliore possibile''?

Ricordiamo a tal proposito la formula di Taylor per $n=1$, prendendo $x^0 \in \mathbb{R}$ e di conseguenza $\mathrm{B}(x^0,\,r_0)=(x^0-r_0,\,x^0+r_0)$.

\begin{thm}[Formula di Taylor con resto secondo Lagrange ($n=1$)]
Sia $f \in C^{m+1}((x^0-r_0,\,x^0+r_0))$. Definiamo
$$
P_m(f,\,x^0)(x) = \sum_{h=0}^{m} \frac{f^{(h)}(x^0)}{h!}(x-x^0)^h
\qquad (x \in \mathbb{R})
$$
Tale $P_m$ è detto \emph{polinomio m-esimo di Taylor di $f$}. Allora:
\begin{center}
$\mathrm{(FT_m)}$
\hfill
$\displaystyle f(x) = P_m(f,\,x^0)(x)+R_m(x,\,x^0)$
\hfill \null \\
\end{center}
con
\begin{center}
$\mathrm{(RL_m)}$\footnote{Resto secondo Lagrange}
\hfill
$\displaystyle R_m(x,\,x^0) = \frac{f^{(m+1)}(\overline{x})}{(m+1)!}(x-x^0)^{m+1}$
\hfill \null \\
\end{center}
dove $\overline{x}$ è un punto opportuno tra $x^0$ e $x$. In particolare, vale che
\begin{center}
$\mathrm{(RP_m)}$\footnote{Resto secondo Peano}
\hfill
$\displaystyle \exists \, \lim_{x \rightarrow x^0} \frac{R_m(x,\,x^0)}{|x-x^0|^m} = 0$
\hfill \null \\
\end{center}
\end{thm}

Ritorniamo ora al caso in $n$ variabili, e prendiamo una funzione $f \in C^m(\mathrm{B}(x^0,\,r_0))$ (ossia $f : \mathrm{B}(x^0,\,r_0) \rightarrow \mathbb{R}$). Ricordiamo che, se $\alpha = (\alpha_1,\ldots,\alpha_n) \in \mathbb{N}^n$, allora $|\alpha| \doteqdot \alpha_1 + \ldots + \alpha_n$. Nel caso $|\alpha| \leq m$, possiamo definire
$$
D^{\alpha} f(x) = \frac{\partial^{|\alpha|}f}{\partial x_1^{\alpha_1} \cdot \ldots \cdot \partial x_n^{\alpha_n}}(x) \qquad \forall \, x \in \mathrm{B}(x^0,\,r_0)
$$

Definiamo poi
$$
\alpha! \overset{def}{=} \alpha_1! \cdot \alpha_2! \cdot \ldots \cdot \alpha_n!
$$
e
$$
(x-x_0)^{\alpha} \overset{def}{=} (x_1-x_1^0)^{\alpha_1} \cdot (x_2-x_2^0)^{\alpha_2} \cdot \ldots \cdot (x_n-x_n^0)^{\alpha_n}
$$
se $x=(x_1,\ldots,x_n)$ e $x^0=(x_1^0,\ldots,x_n^0)$.

Vale allora il seguente teorema.

\begin{thm}[Formula di Taylor]
Sia $f \in C^{m+1}(\mathrm{B}(x^0,\,r_0))$. Definiamo
$$
P_m(f,\,x^0)(x) = \sum_{\overset{\text{\scriptsize $\alpha \in \mathbb{N}^n$}}{|\alpha| \leq m}}^{m} \frac{D^{\alpha} f(x^0)}{\alpha!}(x-x^0)^{\alpha}
\qquad (x \in \mathbb{R}^n)
$$
Tale $P_m$ è detto \emph{polinomio m-esimo di Taylor di $f$}. Allora:
\begin{center}
$\mathrm{(FT_m)}$
\hfill
$\displaystyle f(x) = P_m(f,\,x^0)(x)+R_m(x,\,x^0) \qquad \forall x \in \mathrm{B}(x^0,\,r_0)$
\hfill \null \\
\end{center}
con
\begin{center}
$\mathrm{(RL_m)}$
\hfill
$\displaystyle R_m(x,\,x^0) = \sum_{\overset{\text{\scriptsize $\alpha \in \mathbb{N}^n$}}{|\alpha| = m+1}}^{m} \frac{D^{\alpha} f(\overline{x})}{\alpha!}(x-x^0)^{\alpha}$
\hfill \null \\
\end{center}
dove $\overline{x}$ è un punto opportuno nel segmento $x^0x \doteqdot \lbrace tx + (1-t)x^0 : 0 \leq t \leq 1 \rbrace$. In particolare, vale che
\begin{center}
$\mathrm{(RP_m)}$
\hfill
$\displaystyle \exists \, \lim_{x \rightarrow x^0} \frac{R_m(x,\,x^0)}{||x-x^0||_n^m} = 0$
\hfill \null \\
\end{center}
\end{thm}


\begin{obs}
Supponiamo che valga $(RL_m)$ per $f$ e proviamo che vale $(RP_m)$.
$$|R_m(x,\,x^0)| \leq$$
$$\leq \sum_{|\alpha| = m+1} \frac{\lvert D^{\alpha} f(\overline{x}) \rvert}{\alpha!} \lvert (x-x^0)^{\alpha} \rvert = $$
$$= \sum_{|\alpha| = m+1} \frac{\lvert D^{\alpha} f(\overline{x}) \rvert}{\alpha!} \lvert x_1-x^0_1 \rvert^{\alpha_1} \ldots \lvert x_n-x^0_n \rvert^{\alpha_n} \leq $$
$$ \overset{\footnotemark}{\leq} \sum_{|\alpha| = m+1} \frac{\lvert D^{\alpha} f(\overline{x}) \rvert}{\alpha!} \lvert\lvert x-x^0 \rvert\rvert^{\alpha_1}_n \ldots \lvert\lvert x-x^0 \rvert\rvert^{\alpha_n}_n = $$ \footnotetext{In un triangolo rettangolo ogni cateto è sempre minore dell'ipotenusa!}
$$ \overset{\footnotemark}{=} \sum_{|\alpha| = m+1} \frac{\lvert D^{\alpha} f(\overline{x}) \rvert}{\alpha!} \lvert\lvert x-x^0 \rvert\rvert^{|\alpha|}_n = $$ \footnotetext{$|\alpha| = \alpha_1 + \ldots + \alpha_n$}
$$ = \lvert\lvert x-x^0 \rvert\rvert^{m+1}_n F(x)$$
dove $\displaystyle F(x) = \sum_{|\alpha| = m+1} \frac{\lvert D^{\alpha} f(\overline{x}) \rvert}{\alpha!}, \qquad x \in \mathrm{B}(x^0,\,r_0)$
$$
\Updownarrow
$$
$$
\frac{|R_m(x,\,x^0)|}{\lvert\lvert x-x^0 \rvert\rvert^{m}_n} \leq \lvert\lvert x-x^0 \rvert\rvert_n F(x)
$$
Ora, passando a limite, $\lvert\lvert x-x^0 \rvert\rvert_n$ va a $0$, ma cosa accade a $F(x)$? Poiché le derivate di $f$ sono continue fino a $m+1$, la funzione $F$ è continua, ed è quindi facile provare che è limitata. In particolare, vale che
$$
\lim_{x \rightarrow x^0} F(x) = \sum_{|\alpha| = m+1} \frac{\lvert D^{\alpha} f(x^0) \rvert}{\alpha!}
$$
Infatti, prendendo $\overline{x} = \overline{t}x+(1-\overline{t})x^0$ per un certo $0 \leq \overline{t} \leq 1$, vale:
$$
\lim_{x \rightarrow x^0} F(x) =$$
$$= \lim_{x \rightarrow x^0} \sum_{|\alpha| = m+1} \frac{\lvert D^{\alpha} f(\overline{x}) \rvert}{\alpha!} = \lim_{x \rightarrow x^0} \sum_{|\alpha| = m+1} \frac{\lvert D^{\alpha} f(\overline{t}x+(1-\overline{t})x^0) \rvert}{\alpha!} = \sum_{|\alpha| = m+1} \frac{\lvert D^{\alpha} f(x^0) \rvert}{\alpha!}
$$
dove nell'ultimo passaggio si è usata la continuità delle derivate successive, portando il limite dentro l'argomento della funzione.
\end{obs}

\begin{proof}
(Formula di Taylor, $n=2$)\\
Fissiamo $x=(x_1,\,x_2) \in \mathrm{B}(x^0,\,r_0)$ e $x^0=(x_1^0,\,x_2^0)$. Sia $F : [-1,\,1] \rightarrow \mathbb{R}$, $F(t) \doteqdot f(x^0 + t(x-x^0))$. In particolare, dalla regola di derivazione di una funzione composta, $F \in C^{m+1} ([-1,\,1])$.\\
(Idea della dimostrazione: $F(1)=f(x)$)\\
Applichiamo a $F$ la formula di Taylor (per una variabile) con punto iniziale $t^0 = 0$:
\begin{center}
$\mathrm{(1)}$
\hfill
$\displaystyle F(t) = \sum_{k=0}^{m} \frac{F^{(k)}(0)}{k!}t^k + R_m(t,\,0) \qquad \forall t \in [-1,\,1]$
\hfill \null \\
\end{center}
dove 
$$R_m(t,\,0) = \frac{F^{(m+1)}(\overline{t})}{(m+1)!}t^{m+1}$$
Calcoliamo le derivate $F^{(k)}(t)$ in termini di $D^{\alpha}f$, tenendo conto che $F = f \circ g$ dove $g(t) = x^0+t(x-x^0)$, e quindi $F'(t) = \nabla f(g(t)) \bullet g'(t)$.
\begin{enumerate}[labelindent=\parindent,leftmargin=*,label=\underline{\textnormal{$k=\arabic*$}}:,start=0]
\item $\displaystyle F^{0}(t)=f(x^0+t(x-x^0))$
\item $\displaystyle F'(t) = \nabla f(x^0+t(x-x^0)) \bullet (x-x^0) = \sum_{j=1}^2 \frac{\partial f}{\partial x_j} (x^0+t(x-x^0))(x_j-x_j^0)$
\item Come prima, consideriamo ogni addendo $F'_j$ di $F'$ una funzione $\displaystyle F'_j =  (x_j-x_j^0)\frac{\partial f}{\partial x_j} \circ g$. Di conseguenza:\\
$\displaystyle F''(t) = \sum_{j=1}^2 \left ( (x_j-x_j^0) \nabla \left( \frac{\partial f}{\partial x_j} (x^0+t(x-x^0)) \right) \right) \bullet (x-x^0)=$\\
$\displaystyle =
\sum_{j=1}^2
\left( 
\sum_{i=1}^2 \frac{\partial^2 f}{\partial x_i \partial x_j} (x^0+t(x-x^0))(x_j-x_j^0)
\right) 
(x_i-x_i^0) = $\\
$\displaystyle =
\sum_{i=1,\,j=1}^2
\frac{\partial^2 f}{\partial x_i \partial x_j} (x^0+t(x-x^0))(x_j-x_j^0)(x_i-x_i^0)$
\item[\underline{\textnormal{$k$}}:] $\displaystyle \mathrm{(2)}$\\
$ \displaystyle F^{(k)}(t) = 
\sum_{i_1=1,\ldots,i_k=1}^2
\frac{\partial^k f}{\partial x_{i_1} \ldots \partial x_{i_k}}(x^0+t(x-x^0))
(x_{i_1}-x_{i_1}^0) \ldots (x_{i_k}-x_{i_k}^0)$\\
$\displaystyle \forall t \in [-1,\,1]$
\end{enumerate}

Fissiamo un multi-indice $\alpha = (\alpha_1,\,\alpha_2)$ tale che $|\alpha| = \alpha_1+\alpha_2 = k$ con $0 \leq k \leq m+1$, e determiniamo il numero di volte che
\begin{center}
$\mathrm{(3)}$
\hfill
$\displaystyle D^{\alpha} f (x^0 + t(x-x^0)) = \frac{\partial^k f}{\partial x_{i_1} \ldots \partial x_{i_k}} (x^0 + t(x-x^0))$
\hfill \null \\
\end{center}
per una opportuna funzione $i : \lbrace 1,\,k \rbrace \longrightarrow \lbrace 1,\,2 \rbrace$ quando $i_j \overset{def}{=} i(j)$

Tenendo presente il teorema di inversione dell'ordine delle derivate parziali, vale $\mathrm{(3)}$ se e solo se:
\begin{center}
$\mathrm{(4)}$
\hfill
$\displaystyle 
\begin{cases}
i_j = 1 & \text{per un numero di volte $\alpha_1$}\\
i_j = 2 & \text{per un numero di volte $\alpha_2$}\\
\end{cases}
$
\hfill \null \\
\end{center}
(dove ricordiamo che $j \in \lbrace 1,\,k \rbrace$).

Andando a contare il numero di volte che si verifica $\mathrm{(3)}$, si prova per induzione (esercizio) che tale numero è pari a $\displaystyle \frac{k!}{\alpha_1!\, \alpha_2!}$. Utilizzando questo risultato insieme a $(2)+(3)+(4)$, segue che
\begin{flushleft}
$\displaystyle F^{(k)}(t) =$\\
$\displaystyle = \sum_{i_1=1,\ldots,i_k=1}^2
\frac{\partial^k f}{\partial x_{i_1} \ldots \partial x_{i_k}}(x^0+t(x-x^0))
\underbrace{(x_1-x_1^0)^{\alpha_1} (x_2-x_2^0)^{\alpha_2}}_{\equiv (x-x^0)^{\alpha}}=$\\
$\displaystyle = \sum_{|\alpha=k|}
\frac{k!}{\alpha_1!\, \alpha_2!} D^{\alpha}f(x^0+t(x-x^0))
(x-x^0)^{\alpha}$
\end{flushleft}
Applichiamo ora la $\mathrm{(1)}$ con $t=1$:
\begin{flushleft}
$\displaystyle f(x) = F(1) =$\\
$\displaystyle = \sum_{k=0}^m \sum_{|\alpha|=k} \frac{D^{\alpha}f(x^0)}{\alpha!}(x-x^0)^{\alpha} + \sum_{|\alpha|=m+1} \frac{D^{\alpha}f(x^0+t(x-x^0))}{\alpha!}(x-x^0)^{\alpha} \doteqdot $\\
$\displaystyle \doteqdot P_m(f,\,x^0)(x) + R_m(x,\,x^0)$
\end{flushleft}
\end{proof}

\begin{obs}
Si può provare che, se $f \in C^m(\mathrm{B}(x^0,\,r_0))$, vale ancora $\mathrm{(FT_m)}$ per una opportuna funzione resto $R_m(x,\,x_0) : \mathrm{B}(x^0,\,r_0) \rightarrow \mathbb{R}$ tale che:
\begin{center}
$\mathrm{(RP_m)}$
\hfill
$\displaystyle \exists \, \lim_{x \rightarrow x^0} \frac{R_m(x,\,x^0)}{||x-x^0||_n^m} = 0$
\hfill \null \\
\end{center}
\end{obs}

\begin{obs}
$$f(x) = o(g(x)) \qquad (x \rightarrow x^0)$$
$$\Updownarrow$$
\begin{enumerate}[labelindent=\parindent,leftmargin=*,label=\textnormal{(\arabic*)},start=1]
\item $\displaystyle \exists \, \lim_{x \rightarrow x^0} g(x) = 0$
\item $\displaystyle  g(x) \neq 0 \qquad \forall \, x \in (x^0-r_0,\,x^0+r_0) \setminus \lbrace x^0 \rbrace$
\item $\displaystyle \lim_{x \rightarrow x^0} \frac{f(x)}{g(x)} = 0$
\end{enumerate}
\end{obs}

\subsection{Applicazione al calcolo dei limiti}
\begin{example}[i]
$$\lim_{x \rightarrow 0} \frac{\sin(x)}{x} = 1$$
\end{example}
\begin{proof}
La formula di Taylor al \RNum{2} ordine per la funzione $\sin(x)$ è:
$$\sin(x) = x - \frac{x^3}{6} + R_3(x,\,0) \qquad \text{dove} \qquad R_3(x,\,0) = \frac{f^{(h)}(\overline{x})}{h!}x^h = o(x^3)$$
per un opportuno $\overline{x}$ tra $0$ e $x$. Quindi:
$$\frac{\sin(x)}{x} = 1 - \frac{x^2}{6} + o(x^2)$$
A questo punto, il limite diventa banale.
\end{proof}

\begin{example}[ii]
$$\lim_{x \rightarrow 0} \frac{x-\sin(x)}{x^3} = \frac{1}{6}$$
\end{example}
\begin{proof}
Se avessi sviluppato all'ordine \RNum{1}:
$$\sin(x) = x + o(x)$$
$$\frac{x-\sin(x)}{x^3} = \frac{o(x)}{x^3} \overset{???}{\longrightarrow}$$
A questo punto, mi devo fermare! Avrei quindi dovuto sviluppare di più. Uno dei vantaggi della notazione ``$o$'', a differenza di quella delle equivalenze asintotiche, è proprio il fatto che rende chiaro l'ordine fino al quale sviluppare.
\end{proof}


\subsection{Caso $m=2$}

\begin{definition}
Sia $f \in C^2(A)$, $A \subseteq \mathbb{R}^n$ aperto. Si chiama \emph{matrice hessiana} in un punto $x^0 \in A$ fissato la matrice quadrata e simmetrica (per Schwarz) di ordine $n$:
$$
H(f)(x^0) = \left( \frac{\partial^2f}{\partial x_i \partial x_j} \right)_{i,\,j = 1,\ldots,n}
$$
Ad esempio, nel caso $n = 2$,
$$
H(f)(x^0) = \left(
\begin{array}{cc}
\displaystyle \frac{\partial^2f}{\partial x_1^2} (x^0) & \displaystyle \frac{\partial^2f}{\partial x_1 \partial x_2} (x^0)\\
\displaystyle \frac{\partial^2f}{\partial x_2 \partial x_1} (x^0) & \displaystyle \frac{\partial^2f}{\partial x_2^2} (x^0)\\
\end{array}
\right)_{2 \times 2}
$$
\end{definition}

\begin{obs}[importante]
Supponiamo che $m=2$. Se $f \in C^2(\mathrm{B}(x^0,\,r_0))$, allora vale che:
$$f(x) = P_2(f,\,x^0)(x) + R_2(x,\,0)$$
con
\begin{flushleft}
$\displaystyle P_2(f,\,x^0) = $
$\displaystyle = \sum_{|\alpha| < 2} \frac{D^{\alpha}(f(x^0))}{\alpha!}(x-x^0)^{\alpha}=$
$\displaystyle = \sum_{|\alpha| = 0} \frac{D^{\alpha}(f(x^0))}{\alpha!}(x-x^0)^{\alpha} +
\sum_{|\alpha| = 1} \frac{D^{\alpha}(f(x^0))}{\alpha!}(x-x^0)^{\alpha}+
\sum_{|\alpha| = 2} \frac{D^{\alpha}(f(x^0))}{\alpha!}(x-x^0)^{\alpha} = $
$f(x^0) + df(x^0)(x-x^0) + \sum_{|\alpha| = 2} \frac{D^{\alpha}(f(x^0))}{\alpha!}(x-x^0)^{\alpha}$
\end{flushleft}
Si può facilmente verificare (basta fare i conti!) che:
$$\sum_{|\alpha| = 2} \frac{D^{\alpha}(f(x^0))}{\alpha!}(x-x^0)^{\alpha} = 
\frac{1}{2} \left( H(f)(x^0)(x-x^0)^T \bullet (x-x^0) \right)$$
Concludendo, se $f \in C^2(\mathrm{B}(x^0,\,r_0))$, vale che:
\begin{center}
$\mathrm{(FT_2)}$
\hfill
$\displaystyle f(x) = f(x^0)+df(x^0)(x-x^0)+\frac{1}{2} \left( H(f)(x^0)(x-x^0)^T \bullet (x-x^0) \right)
+ R_2(x,\,x^0)$
\hfill \null \\
\end{center}
con
$$
\exists \lim_{x \rightarrow x^0} \frac{R_2(x,\,x^0)}{||x-x^0||_2^2}
$$
\end{obs}











