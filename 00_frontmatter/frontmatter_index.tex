\frontmatter
\renewcommand{\thepage}{\roman{page}}
\maketitle

\begin{footnotesize}
\null
  \begin{tiny}
  \texttt{\textcolor{gray}{Versione: \version\today-\microversion}}
  \end{tiny}
\vfill

\begin{figure}[hb]
\centering
\includegraphics[width=4cm]{00_frontmatter/figures/cc-by-sa}
\caption[Licenza]{Logo CC BY-SA}
\end{figure}

\emph{Questo documento è rilasciato in tutte le sue parti sotto licenza Creative Commons } \href{http://creativecommons.org/licenses/by-sa/3.0/deed.it}{CC BY-SA}\emph{. Per eventuali file non qui specificati che possiedono una licenza differente, fai riferimento agli stessi.}

CC BY-SA\emph{ significa che puoi redistribuire o modificare questo testo, a patto che usi la stessa licenza, che venga riconosciuta la paternità dell'opera e che eventuali modifiche o revisioni non ufficiali siano messe in evidenza in modo chiaro, per preservare l'identità dell'opera originale (per maggiori dettagli, consultare ad esempio \href{http://it.wikipedia.org/wiki/Licenze_Creative_Commons\#Attribuzione}{Wikipedia}). I metadati sulla licenza sono inoltre incorporati nel documento stesso. \`E bene sottolineare che l'opera è protetta dai diritti d'autore, sebbene la sua modifica sia libera.}

\emph{Parte del layout e dell'organizzazione è ispirata a }PhDtemplateLaTeX\emph{, per cui ringrazio il suo autore.}

\emph{Proseguendo nella lettura, accetti automaticamente le condizioni di rilascio.}

\vskip 10pt

Contatti:  \href{mailto:secli.matteo@gmail.com}{secli.matteo@gmail.com}
\end{footnotesize}

\tableofcontents
%\listoftables
%\listoffigures


\mainmatter
\renewcommand{\thepage}{\arabic{page}}\setcounter{page}{1}